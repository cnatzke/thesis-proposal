\documentclass[jon_ringuette_thesis_proposal.tex]{subfiles}
\begin{document}

    \chapter{Conclusions}
    The TITAN EBIT is uniquely suited to make the NEEC measurement due to its ability to precisely control ionization and electron energy.
    In addition the TITAN ISAC facility enables the TITAN EBIT to have access to unstable beam greatly increasing the number of candidates for a NEEC measurement.
    This puts the TITAN EBIT in an excellent position to clear up the current 9 orders of magnitude discrepancy between the NEEC theoretical cross-section \cite{Wu2019} and the Agronne lab experiment \cite{Chiara2018}.
    With the TITAN EBIT decay spectroscopy setup it should be possible to clear up at least 3 orders of magnitude difference in this measurement with future work going even further towards direct observation of NEEC.

    The current EBIT decay spectroscopy setup is nearly ready for commissioning.
    With 2 of the 6 germanium detectors installed on the EBIT it has been possible to test all plumbing, cryogenic and electrical system to ensure their readiness.
    The detectors have been tuned and the DAQ, HVPS control systems, and the LN2 auto-fill systems have all been thoroughly tested.
    Simulations have been completed for the experiment from charge breeding using pyCBSIM all the way through optimization of trapping times and determination of experimental NEEC rates culminating in a full GEANT4 simulation (\ref{fig:neec_2day_500ma_simulation}).
    Given all the work done the TITAN EEC approved this experiment to run with High Priority in 2021.
    However, since then the TITAN EBIT has experienced multiple system failures and the TITAN EBIT team is currently in the process of having the electron gun replaced over the course of the next year and is beyond the scope of this work.

    Work towards further beam purification using the MR-TOF-MS as well as further optimizations of the EBIT trapping cycles is ongoing with a full analysis of a new highly optimized MR-TOF-MS measurement of $^{35}Mg$ coming shortly.
    %At this time it is thought that the initial case of $^{129m}Sb$ is the best candidate at the TRIUMF ISAC facility due to its long lifetime and small gap between isomeric energy levels that should be achievable in the TITAN EBIT.
\end{document}