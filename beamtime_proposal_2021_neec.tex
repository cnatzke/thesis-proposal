% eec_nrp_dspr.tex       written by M. Comyn       last updated 23-05-2012 by NM

% Use this LaTeX file to submit your TRIUMF EEC New Proposal
% Detailed Statement of Proposed Research

% NOTE:
% You must enter your experiment number between the braces of the
% \setcounter{expnum}{}
% command located at the foot of this file, prior to entering your text.

%##########################################################################

\documentclass[12pt]{article}
\usepackage{verbatim,latexsym,multicol,multirow,graphicx,rotating,subcaption,url,doi}
\usepackage[sort&compress,numbers,plainnat]{natbib}
\bibliographystyle{apsrev4-1}


% \usepackage{amsmath}
%\usepackage{amssymb}
%\usepackage{dcolumn}
%\usepackage{longtable}
%\usepackage{pifont}

% Use Michel Goossens' dense lists Itemize, Enumerate and Description

\let\Otemize =\itemize
\let\Onumerate =\enumerate
\let\Oescription =\description
% Zero the vertical spacing parameters
\def\Nospacing{\itemsep=0pt\topsep=0pt\partopsep=0pt\parskip=0pt\parsep=0pt}
% Redefine the environments in terms of the original values
\newenvironment{Itemize}{\Otemize\Nospacing}{\endlist}
\newenvironment{Enumerate}{\Onumerate\Nospacing}{\endlist}
\newenvironment{Description}{\Oescription\Nospacing}{\endlist}

\newcounter{expnum}

\renewcommand{\thebibliography}[1]{\list
 {\arabic{enumi}.}{\settowidth\labelwidth{[#1]}\leftmargin\labelwidth
 \advance\leftmargin\labelsep
 \usecounter{enumi}}
 \renewcommand{\newblock}{\hskip .11em plus .33em minus -.07em}
 \itemsep=0pt\topsep=0pt\partopsep=0pt\parskip=0pt\parsep=0pt
 \sloppy
 \sfcode`\.=1000\relax}
\let\endthebibliography=\endlist

\makeatletter
\renewcommand{\ps@plain}{%
 \renewcommand{\@oddhead}{\sffamily\bfseries
 \hspace{-2mm}%
 \begin{tabular}[t]{|p{180mm}|}\hline
  {\footnotesize
  TRIUMF EEC New Proposal
  \hspace{\fill}
  Detailed Statement of Proposed Research for Experiment \#: \theexpnum}\\ \hline
 \rule[-242mm]{0mm}{0mm}\\ \hline
 \end{tabular}}
 \renewcommand{\@evenhead}{\@oddhead}
 \renewcommand{\@oddfoot}{\sffamily\hfil\thepage\hfil}
 \renewcommand{\@evenfoot}{\@oddfoot}}
\makeatother

\textwidth      180mm
\textheight     240mm
\topmargin      -20mm
\oddsidemargin   -7mm
\flushbottom
\frenchspacing

\pagestyle{plain}

\begin{document}

\footnotesize\sffamily

%\noindent
%\textbf{The EEC committees strongly recommend that you limit your
%submissions, including figures and tables, to no more than 4 pages for the
%MMSEEC or 10 pages for the SAPEEC. \ The following information should be
%included:}%

%\vspace*{-2mm}

\scriptsize\sffamily

%\begin{Itemize}
%\item[(a)]
%\textbf{Scientific value of the experiment:}
%Describe the importance of the experiment and its relation to previous
%work and to theory.  All competitive measurements at other laboratories
%should be mentioned.  Include examples of the best available theoretical
%calculations with which the data will be compared.
%\item[(b)]
%\textbf{Description of the experiment:}
%Techniques to be used, scale drawing of the apparatus, measurements to be
%made, data rates and background expected, sources of systematic error,
%results and precision anticipated.  Compare this precision with that
%obtained in previous work and discuss its significance in regard to
%constraining theory.
%\item[(c)]
%\textbf{Experimental equipment:}
%Describe the purpose of all major equipment to be used.
%\item[(d)]
%\textbf{Readiness:}
%Provide a schedule for assembly, construction and testing of equipment.
%Include equipment to be provided by TRIUMF. For secondary beam for ISAC, provide information on established yields of the isotope of interest as well as the established isobaric contaminants form the same target/ion-source combination.
%\item[(e)]
%\textbf{Beam time required:}
%State in terms of number of 12-hour shifts.  Show details of the beam time
%estimates, indicate whether prime-user or parasitic time is involved, and
%distinguish time required for test and adjustment of apparatus.
%\item[(f)]
%\textbf{Data analysis:}
%Give details and state what data processing facilities are to be provided
%by TRIUMF.
%\end{Itemize}

\footnotesize\sffamily
\normalsize\rmfamily

%##########################################################################

\setcounter{expnum}{2128} %Enter your experiment number between the braces

%##########################################################################

% Enter Detailed Statement of Proposed Research below:
\begin{center}\textit{\footnotesize This proposal is the follow-up to S1865LOI for which the development of $^{129m}$Sb at ISAC has been demonstrated}\end{center}
\vspace{-20pt}
\section{Scientific Value}
\subsection{Nuclear Excitation by Electronic Processes}
Some common modes of electroweak nuclear decay - such as orbital electron capture (EC) and internal conversion (IC) - proceed through an interaction between the nucleus and bound electrons within the constituent atom.  Analogously, the nucleus can also be \textit{excited} through its interactions with the atomic states if the energies of the accessible nuclear states are low-enough in energy.  As a result, the probabilities of the respective processes are not only influenced by the structure of the initial and final nuclear states, but can also depend strongly on the atomic and environmental conditions in which the nucleus exists.  Nuclear Excitation via Electron Capture (NEEC) is one such process where a free electron is captured from the continuum into an atomic vacancy and simultaneously excites the nucleus to a higher-energy state (Fig.~\ref{NEECProcess} a))~\cite{Gol76}.  This interaction occurs by Coulomb interaction or virtual photon exchange between the electronic and nuclear currents~\cite{Pal06}.  Unlike processes such as broadband photoexcitation of isomers with bremsstrahlung photons, NEEC is a \textit{resonant} process and requires specific energy matching conditions for the excitation to occur - thus making it incredibly selective.  Specifically, the sum of the kinetic energy of the free electron and captured binding energy must correspond to the energy difference between the initial and final nuclear states~\cite{Pal07}. This results in the need for strong atomic charge-state control over the sample, as well as careful case selection of nuclear states that may be compatible with efficient electron recombination.

Since the electron is captured from the environment and must match a resonance condition, the prospect of using NEEC within the context of applied nuclear physics is indeed appealing.  This is due to the fact that NEEC may be particularly efficient at exciting nuclear isomers which subsequently decay through a cascade that circumvents the metastable state and leads to the release of the energy stored as nuclear excitation~\cite{Pal07}.  In the cases which have the possibility to undergo NEEC, a small nuclear excitation ($\Delta E<$10's keV) leads to the release of MeV-scale stored energy from the isomeric state (Fig.~\ref{NEECProcess} b)). For this reason, these types of physical processes have been investigated as one of the most promising sources for possible nuclear battery technologies, which could provide energy densities $\sim10^5$ times greater than those in traditional chemical batteries~\cite{Chi18,Pal07}.   From a more fundamental standpoint, the NEEC process may also play a critical role in extreme astrophysical environments such as those connected to the rapid neutron capture process ($r$ process).

%%%%%%%%%%%%%%%%%%%%%%%%%%
\begin{figure}[!t]
  \centering
  \includegraphics[width=\linewidth]{NEEC_Process_Image.png}
  \caption{\label{NEECProcess}\small A depiction of a) the NEEC excitation process in both the atom (left) and nucleus (right), and b) the nuclear $\gamma$ cascade following the population of the excited state via NEEC.}
\end{figure}
%%%%%%%%%%%%%%%%%%%%%%%%%%

\subsection{NEEC and Nuclear Astrophysics}
%For the slow neutron-capture ($s$) process with $T_6$=250-1000 ($T_6$ is the temperature in million K) this means that the maximum of the MB distribution ($kT=0.0862\times T_6$) is at 22~keV and 86~keV, respectively.  However, the high-energy tail of the distribution allows the thermal population of excited states up to several hundreds of keV.

%ID 15/03/2021 New pimped up text
In astrophysical events, the charge-states of the ions as well as the population of excited states are governed by the electron densities and the temperature of the stellar plasma.  Explosive scenarios like the $r$ process in core collapse supernovae and binary neutron star mergers easily reach temperatures in excess of 1--2~GK, corresponding to a Maxwell-Boltzmann velocity distribution of the participating particles. The most probable energy of the particles in the plasma is then between $kT$=86--172~keV but the high-energy tail of the distribution allows thermal population of states with several 100's of keV excitation energy.

This population of higher-lying states can lead to detours of the reaction flow, causing an acceleration or deceleration of the decay if certain $\beta$-decaying states are bypassed. These so-called \textit{"gateway states"} play a significant role for heavy element nucleosynthesis processes and can lead to wrong conclusions in the interpretation of abundances of stable nuclei when compared to terrestrial conditions.  The population of these gateway states can occur in different processes. They are not only thermally populated by ($\gamma, \gamma'$) but also by inelastic neutron scattering via ($n,n' \gamma$) reactions or even in secondary reactions by exotic decay processes like NEEC and NEET (Nuclear Excitation by Electronic Transitions). Unfortunately, these excitation processes -- with exception of thermal population -- are so far not well investigated or even included in astrophysical simulations or post-processing codes.  In fact, whether NEEC has ever been observed is still a major open question (Section~\ref{discrepancy}).

Several nuclei have been identified where these gateway states can be populated under stellar conditions and need to be considered for a complete understanding of the differences in measured vs. calculated abundances. Examples are
\begin{itemize}
    \item the Cd-In-Sn isotopes in mass region $A$=113--115 where many isomeric states exist and the reactions flows of all three heavy element nucleosynthesis processes ($s$, $r$, $\gamma$) come together~\cite{Nem94}.
    \vspace{-10pt}
    \item long-lived quasi-stable $^{176}$Lu and $^{180}$Ta$^m$ which were originally proposed as $s$-process chronometers but the population of gateway states changes their effective stellar lifetimes drastically~\cite{Car89}.
\end{itemize}


%Another possibility of influencing the decay of certain astrophysically relevant nuclei in a stellar plasma is by ionization. These effects have been described for the bound-state $\beta$-decay of fully-ionized nuclei -- such as $^{187}$Re$^{75+}$ -- which together with $^{187}$Os is used as chronometer to determine the age of the Galaxy (F. Bosch, PRL77, 5190 (1996)). Another example is the partial blocking of the EC of $^{7}$Be in the Sun which leads to a $\approx$2x longer solar lifetime of $^{7}$Be.

Despite the fact that these processes are not included in astrophysical simulations, such exotic decay modes could modify reaction flows, e.g. back to stability in the $r$-process freeze-out phase. In Fig.~\ref{A129-decay} the $A$=129 decay chain starting at $^{129}$Sb back to stability ($^{129}$Xe) under terrestrial conditions is shown.  If the relative NEEC probabilities are indeed as large as reported in Ref.~\cite{Chi18}, this process would have a significant effect.

%%%%%%%%%%%%%%%%%%%%%%%%%%
\begin{figure}[!htb]
  \centering
  \includegraphics[width=0.6\linewidth]{129Sb-decays-NEEC.png}
  \caption{\label{A129-decay}\small{Change in the reaction flow for the $A$= 129 decay chain from $^{129}$Sb up to stability. (Top) Under terrestrial conditions, the majority of the decay from the (19/2$^-$) isomeric state would go through the 11/2$^-$ isomer in $^{129}$Te. (Bottom) If the NEEC process is strong, the (19/2$^-$) isomeric state would be depopulated and decays mainly occur via the low-spin ground-states.} }
\end{figure}
%%%%%%%%%%%%%%%%%%%%%%%%%%

The long-lived $^{129}$I ($t_{1/2}$= 15.7~My) plays a special role in this decay chain since it can be detected in meteorites (presolar grains) and its abundance ratio with $^{247}$Cm in the Early Solar System can be used as direct observational constraint of the last $r$-process event that seeded the presolar nebula~\cite{Ben21}.
%ID: ratio 129I/247Cm is 438 +/- 184 is Early Solar System. Not sure if we should mention that in more detail here...
%Theoretical nuclear physics uncertainties also very large for calculating r-process abundances.
The NEEC of the high-spin isomeric state in $^{129}$Sb can alter the decay path back to stability via bypassing the high-spin isomeric state in $^{129}$Te that has a longer half-life than its respective ground-state before the reaction flow merges back at $^{129}$In. Interestingly, \textit{IF} there was also a negative-parity high-spin isomer in $^{129}$I with similar configuration as in $^{129}$Sb (the configuration of the (19/2$^-$) state is $\pi$g$_{7/2}\otimes\nu$h$_{11/2}\otimes\nu$s$_{1/2}$), the main decay path under terrestrial conditions would bypass the long-lived low-spin ground-state of $^{129}$I and thus produce in astrophysical calculations a much lower abundance of this isotope than in the NEEC case.

%\textbf{To discuss: What would it mean for the branching ratios if the NEEC cross section is as high as Chiara claims? How much \% would still go to the $\beta$-decay and how much would go via the gateway states to the g.s. (kind of contributing to the IT fraction)?? Kyle says it is 2\%?}


%In these situations, the electron densities (and atomic charge states) are sufficient to meet the NEEC energy matching requirements discussed above.  Astrophysical studies usually consider the equilibrated, thermalized state population. In doing so, the equilibration via NEEC, internal conversion (IC), and all other pairs of processes are linked by a detailed balance that is implicitly included. The only difference upon including or not including the NEEC/IC process for equilibrium is the required time. As soon as equilibrium is reached, specific processes within detailed balance of processes involving photons, electrons, electron collisions, etc. no longer play a role.

%In the cases where the astrophysical environment drops out of equilibrium\footnote{For example, during the freeze-out phase of the $r$-process when the temperature and neutron density decrease but are still high enough that NEEC could spontaneously occur.}, however, these specific processes become important again and may also play a significant role in the survival rate of nuclei.  If NEEC does indeed occur, nuclei in isomeric states would be excited to shorter-lived states that would decay at a much faster rate than under terrestrial conditions~\cite{Gos04}, which could affect the reaction flow and lead to detours. The consequence is the production of nuclei that can otherwise not be reached under the given conditions, similar to "new" branching points in the $s$-process like for the bound-state $\beta$-decay of $^{163}$Dy and $^{187}$Re.  These processes could also alter the reaction flow (and thus the isotopic abundances), which can lead to misleading interpretations of stable isotope patterns, as measured in e.g. presolar grains.

%%%%%%%%%%%%%%%%%%%%%%%%%%
\begin{comment}
\begin{figure}[!t]
  \centering
  \includegraphics[width=0.9\linewidth]{NEEC-A130region.png}
  \caption{\label{NEEC-A130}\small Possible NEEC cases around $N$=82 isotope $^{130}$Cd are displayed on the chart with stars. The freeze-out path of the $r$-process material in the $A$=129 isotopic chain could be hampered due to different detours through NEEC transitions.  In this LOI, we propose to make our first measurement on $^{129}$Sb which is shown with red star.}
\end{figure}
\end{comment}
%%%%%%%%%%%%%%%%%%%%%%%%%%

%Two astrophysically relevant mass regions are particularly affected by this possibility, which show a significant accumulation of isotopes that satisfy the physical NEEC conditions:
%\begin{itemize}
%\item $A=115$:  $^{113}$Cd and $^{112,114}$In are isotopes where NEEC could occur.  Here, the reaction flow of all three production processes for heavy nuclei ($s$ and $r$ process, and $\gamma$ process for neutron-deficient nuclei) are intertwined. Possible detours of isomeric states via NEEC will further complicate the understanding of this intricate reaction flow network.
%\item $A=130$:  The neutron-rich region around the important $N$=82 $r$-process waiting-point nucleus $^{130}$Cd which is mainly responsible for the second $r$-process abundance peak in the solar abundance curve at $A$=130.  This region is of particular interest for this LOI.
%In Fig.~\ref{NEEC-A130} the NEEC cases on both sides of the valley of stability are shown, together with white dashed lines that indicate schematically the decay back to stability for $A$=128-132 during the freeze-out phase of the $r$-process (neglecting detours from $\beta$-delayed neutron emission).
%\end{itemize}

%The $A$=129 isotopic chain has three relevant isobars, $^{129}$In, $^{129}$Sb, and $^{129}$Xe. The isomeric states from where the NEEC can be induced are all negative-parity high-spin states (23/2$^-$, 19/2$^-$, and 11/2$^-$, respectively) which, apart from the 11/2$^-$ isomer in $^{129}$In, would not be populated by the $\beta$-decay chain from the $r$-process isotopes, e.g. $^{129}$Ag (ground-state 9/2$^+$). These high-spin states can however be populated directly by the repeated fission recycling that occurs once the $r$-process material has reached the neutron-rich actinide region beyond $A$=260.

%Neutron-star merger calculations (see e.g. Ref. \cite{Korobkin2012}) show a robust abundance pattern almost independent of the mass of the two neutron stars, populating the 2nd $r$-process peak at $A$=130 via fission. This neutron-rich material from the neutron star merger is quickly ejected and cools. For these fission fragments decaying back to stability, the NEEC process significantly slows the decay through $^{129}$Sb since the bypassed isomer is shorter-lived (17.7~min) than the ground state (4.3~h).

\subsection{\label{discrepancy}Discrepancy Over First Reported Observation of NEEC in $^{93m}$Mo}
Despite the fact that the NEEC process was originally predicted in the mid 1970's by Goldanskii and Namiot~\cite{Gol76}, the resonance energy-matching requirement makes NEEC difficult to stimulate in the laboratory.  In fact, following decades of efforts towards the demonstration of NEEC, the first claimed observation was reported in 2018 in $^{93m}$Mo using the isomer depletion method in a beam-based experiment at Argonne National Laboratory~\cite{Chi18}.  In this method, the possible stimulation of NEEC is not directly observed, but rather inferred via the depletion of the isomeric state.  The experiment used a high-energy ion beam that was passed through a thin target to prepare the Mo ions in an average charge state between about $q=32^+-36^+$ - suitable for atomic capture of electrons to stimulate NEEC. The highly ionized beam then interacted with the reaction target which statistically reduced the energy of the ions while simultaneously providing electrons that can be captured back into the vacated atomic orbitals.  The NEEC process occurred when one of the combinations of the charge state of the $^{93m}$Mo ion and the effective electron kinetic energy released enough energy to match 4.86~keV needed to excite the nucleus from the isomer to the intermediate state.

The unexpectedly large NEEC probability reported in Ref.~\cite{Chi18} ($P_{exc}=0.010(3)$) led to a detailed theoretical evaluation of the NEEC rate estimates in $^{93m}$Mo by the Heidelberg group~\cite{Pal19} to determine if indeed NEEC was the reason for the observed isomer depletion.  The theoretical work evaluated NEEC probabilities for the dominant 648 recombination channels as a function of charge state and ion energy in the $^{93m}$Mo experiment.  The resulting total NEEC stimulation probability was found to be $P_{exc}\approx5\times10^{-11}$ - nearly 9 orders of magnitude smaller than the observed probability in Ref.~\cite{Chi18}.  The massive discrepancy between these two values is still unknown, however the authors of Ref.~\cite{Pal19} suggest that the observed isomer depletion may be due to non-resonant nuclear excitations resulting from the relatively large energies involved in the in-beam experimental environment.  This remains an open question.

Here, we present a method to search for NEEC in $^{129}$Sb using an Electron Beam Ion Trap (EBIT) at energies below those required for non-resonant excitation to test the underlying theory of NEEC.  The EBIT provides a high level of control of the atomic charge state and electron energies required for the NEEC process, thus generating a much cleaner environment for study.  Additionally, our decay spectroscopy setup with the TITAN-EBIT~\cite{Lea15} is able to identify NEEC stimulation directly since we are able to selectively tune (and de-tune) the isomeric stimulation {\it in-situ} and observe the correlated nuclear $\gamma$ cascade.  \textbf{The proposed experiment aims to resolve the apparent 9 order of magnitude discrepancy between theoretical estimates~\cite{Pal19} and the first claimed NEEC observation~\cite{Chi18} by testing the underlying NEEC theory.}

\section{Experimental Description}
\begin{figure}[t!]
  \centering
  \includegraphics[width=\linewidth]{NEECDecayScheme.png}
  \caption{\label{NEEC}\small The low-lying state level scheme for $^{129}$Sb.  Shown are the first and second isomeric states at 1851~keV and 1861~keV, respectively, between which the NEEC process will be stimulated.  The signal of the NEEC stimulation is the detection of the weakly populated 732~keV, 700~keV, and/or 1161~keV $\gamma$ rays which uniquely follow the de-excitation of the 1861~keV $^{129m2}$Sb state.  The lower-right inset shows the $\gamma$ lines that follow the $\beta^-$ decay of $^{129m1}$Sb.}
\end{figure}
\subsection{\label{129SbCase}NEEC of the First Isomeric State in $^{129}$Sb}
The nuclear structure of $^{129}$Sb provides a good first case\footnote{This was in fact first identified by the GSI-CRYRING group as part of an effort to initiate a more focused NEEC-program~\cite{Let16}} to study controlled NEEC in an EBIT since it contains two isomeric states separated by 9.70(8)~keV~\cite{NNDC}, with clean and unique $\gamma$-ray signatures (Fig.~\ref{NEEC}).  The NEEC stimulation condition to the 2.2~$\mu$s $^{129m2}$Sb state (an $E2$ transition) is fulfilled through the capture of a 6.95~keV electron into the $3p$ atomic vacancy ($BE_M=2.81(1)$~keV~\cite{PalPC}) which is generated by charge-breeding to $^{129}$Sb$^{41+}$ (Ne-like) in the EBIT.  Ionization to Sb$^{41+}$ can be done efficiently in the TITAN-EBIT in its current configuration (Section~\ref{multi}) through so-called ``threshold" charge breeding.  In this technique, we provide an ionization energy just below the energy for a closed atomic shell (Ne-like in this case) and take advantage of the large energy gap to the next ionization state to maximize a single charge state in the trap.  The NEEC cross-section for capture into the $3p$ atomic orbit in $^{129m}$Sb$^{41+}$ is calculated to be $7\times10^{-6}$ b$\cdot$eV~\cite{PalPC}.  The signals of NEEC stimulation result from the decay of $m2$ state, which is dominated by a 9.76(8)~keV IC transition back to the $m1$ state (not observed in this experiment), as well as three unique $\gamma$-rays at 700~keV (BR=7\%), 732~keV (BR=3\%), and 1161~keV (BR=7\%) (Fig.~\ref{NEEC}).

\subsection{Apparatus\label{apparatus}}
\begin{figure}[t!]
  \centering
  \includegraphics[width=0.45\textwidth]{EBIT1_HPGe.png}~~~\includegraphics[width=0.53\linewidth]{sb129_7000_60sec.png}
  \caption{\label{TITAN}\small (left) Cross-sectional view of the EBIT configuration for decay spectroscopy~\cite{Lea15}.  (right) Simulated charge-state distributions as a function of time in the EBIT for an initial e-gun voltage of 6.6(1)~kV for optimized ``threshold" breeding to the required Ne-like charge state ($q=41+$).  The gun voltage is then quickly changed to 7.0(1)~kV at 0.69~s into the cycle to provide the necessary electron resonance energies for NEEC stimulation.  These simulations assume a 0.2~A electron-beam current, $120~\mathrm{\mu m}$ for the electron-beam radius, and a trap pressure of $10^{-11}\mathrm{T}$.}
\end{figure}
The proposed NEEC experiment will be performed using TRIUMF's Ion Trap for Atomic and Nuclear science (TITAN).  We require the use of three of TITAN's ion traps:
\begin{itemize}
\item RFQ: A buffer-gas-filled radio-frequency quadrupole linear Paul trap, used to cool and bunch the $\sim20$~keV ISAC beam to thermal energies, with a small energy-spread.  The cooled beam-bunch is then pulsed and extracted at $\leq$2~keV for transport through the TITAN facility.  The space-charge limit of the RFQ is roughly $10^6e$~\cite{Smi06}.
\item MR-TOF: The Multiple-Reflection Time-of-Flight isobar separator.  In the MR-ToF, different species are separated by their time-of-flight within an isochronous time-of-flight mass analyzer~\cite{Jes15}. The MR-ToF allows for high-resolution separation of isobaric contaminants.  For the work proposed here, the isomeric state is high enough in energy the MR-ToF can efficiently perform \textit{isomeric} separation to provide a pure sample of $^{129m1}$Sb to the EBIT.  The space charge limit of the mass-selective re-trapping RFQ is $10^5-10^6e$.
\item EBIT: An electron-beam ion trap used for charge-breeding~\cite{Lap10} and decay spectroscopy~\cite{Len14,Lea15,Lea15b,Lea15c,Lea17}.  The EBIT is composed of an electron gun, an electron collector, and a cold drift-tube assembly which is thermally coupled to a superconducting magnet.  The space-charge limit of the EBIT is dependent on the electron-beam current, and for our 200 mA e-gun is roughly $10^{10}e$, which translates into a few $10^8$ HCIs stored in the trap.  This e-gun generates an electron beam density of $6\times10^{11}$ cm$^{-3}$ and an energy spread of $\sim0.1$~keV.  The EBIT has 7 access ports that house 6 former 8pi HPGe detectors (with copper and lead shielding) and 1 ultra-LEGe (Fig.~\ref{TITAN} (left)).  The total geometric acceptance of the seven ports depends on the size of the trapped ion cloud, but is roughly 2\% of $4\pi$.  The EBIT photon spectrum is dominated by low-energy ($<5$~keV) X-rays from the atomic de-excitations characteristic of constant charge-state cycling, however this does not limit our HPGe detectors due to the attenuation of these low-energy photons in the thick Al housing material of the cryostat.
\end{itemize}

\subsubsection{Purification of the $^{129m}$Sb State in the MR-ToF}
\begin{figure}[t!]
\begin{center}
\includegraphics[width=0.6\linewidth]{129 beam composition.png}~~~\includegraphics[width=0.35\linewidth]{MRtof_retrapping.png}
\end{center}
\caption{\label{129spectrum} (left) Experimental time-of-flight spectrum in the TITAN MR-ToF from $A=129$ beam delivered from ISAC for a previous $^{129}$In mass measurement experiment.  This spectrum results from ISAC beam using UCx+IGLIS with selective laser tuning to indium, and was thus not optimized for Sb - however known suppression/ionization rates are used here to provide a reliable estimate.  (right) Previous demonstration of the mass-selective retrapping method with the TITAN MR-ToF for separation of a $\sim1.8$~MeV isomer in $^{127}$In$^+$ ions.}
\end{figure}
\begin{comment}
\begin{figure}[t!]
\begin{center}
\includegraphics[width=1\linewidth]{Principle-of-re-trapping.png}
\end{center}
\caption{\label{re-trapping}a) A schematic of the individual stages of an experiment using mass-selective re-trapping of ions in the TITAN MR-ToF. b) Example time-of-flight spectra of $^{40}$Ar$^{+}$ and $^{40}$K$^{+}$ after 161 turns inside the analyzer recorded on a MCP detector. c) Same spectra, but this time the ions underwent a re-trapping operation, before performing undergoing 161 turns for the mass measurement. Figure adopted from~\cite{Dickel2017}.}
\end{figure}
\end{comment}

In order to observe the small anticipated signal resulting from NEEC stimulation, it is advantageous to have the $^{129m1}$Sb state selectively delivered to the TITAN EBIT.  This is required primarily to minimize photon backgrounds in the regions of interest from other decay processes of the radioactive $A=129$ ions.  The MR-ToF allows for high-resolution separation and removal of the isobaric/isomeric contaminants using a technique called ``mass-selective re-trapping".  In this method, the ions are directed towards to the RF injection trap and the ions of interest are mass selectively re-trapped by switching the injection trap quickly from a linear retarding potential to a trapping potential~\cite{Dickel2017}.  By selecting an appropriate re-trapping time, only the ions of interest are captured, while the unwanted ion species are discarded.  In this operation mode, the MR-ToF can efficiently separate the $^{129m1}$Sb state at an excitation energy of 1861~keV from the ground state by their difference in time-of-flight. This mass-selective retrapping for isomeric separation has been demonstrated by TITAN previously for a 1.8~MeV state in $^{127}$In$^+$ ions (Fig.~\ref{129spectrum} (right)) with a resolving power of $7\times10^4$.  Here, we plan to cycle the MR-ToF at 100~Hz while constantly accumulating the separated $^{129m}$Sb$^+$ ions in the re-trapping trap, and extract a single bunch ($\sim2\times10^5$) into the EBIT every minute based on the EBIT cycling discussed below.

%\subsubsection{\label{ChargeState}Charge-State Distributions}
%Due to the nature of these measurements, charge-state identification and selection is critical.  Charge breeding with the EBIT does not produce a single charge state, but rather a distribution with several charge states with varying population probabilities depending on the specific EBIT conditions. However, in cases where there is a large difference in the ionization energy of two successive charge states, i.e. ions that have closed atomic shells, EBIT parameters can be chosen to predominantly populate the threshold charge state. In order to investigate this, EBIT simulations were performed to determine the relative populations of the needed charge state to cause NEEC to occur in Ne-like $^{129}$Sb (Fig.~\ref{129spectrum} (right)).

\subsubsection{\label{multi}EBIT Trapping and Charge Breeding}
In the EBIT, the trapped radioactive ions are axially confined in an electrostatic potential generated by the drift-tube assembly and radially confined by the magnetic field and space-charge potential of the electron beam~\cite{Lap10}.  When trapped, the ions are under constant electron bombardment thus increasing their charge state through electron-impact ionization.  The resulting equilibrium charge-state distribution is defined by the electron-beam current/density/energy, the trapping time, the vacuum in the trap, and decay lifetime.  Here, an electron beam energy of 6.65~keV will be used to perform the threshold charge breeding approach discussed in Section~\ref{129SbCase} to generate an optimized distribution of $^{129m}$Sb$^{41+}$.  For stimulation of NEEC, the electron beam energy is then increased rapidly to the resonance-condition energy of 6.95~keV.  A detailed simulation of the time-dependent trap population of charge states for these conditions, including known ionization and recombination effects is shown in Fig.~\ref{TITAN} (right).

%To overcome the space-charge limit of the upstream components of TITAN (RFQ and MR-ToF), the typical injection scheme must be modified.  Since the space-charge limit of the EBIT is much larger than the RFQ (or MR-ToF), the ISAC beam intensity anticipated for $^{129m}$Sb cannot be fully used during a typical injection/experimentation scheme for decay in the EBIT.  This is of particular concern for this experimental program, as the NEEC excitation rates are typically weak, and require a large number of ions in the trap to observe the process with any statistical significance.  Further, given the long half-life of the first isomeric state (17.7~min), the opportunity to store these ions in the EBIT for several seconds (or longer) is possible without significant losses.

%A method for overcoming this space-charge limit was therefore tested using the $\beta^-$ decay of $^{116}$In by injecting many ion bunches into the EBIT without extraction~\cite{Ros13}.  To achieve this, the inner electrode potential is lowered for first ion-bunch injection and subsequently raised to confine the first ion bunch.  Following this, the injected ion bunch(es) quickly reach $q>2^+$ and remain confined during subsequent injections due to the increased effective potential experienced by the highly charged ions.  The ions are then ejected, and the cycle is repeated.  With the development of the multiple-injection technique outlined in Refs.~\cite{Kla15,Lea15b}, we are able to purify and inject into the EBIT several times per second, while continuously breeding and storing the trapped ions.

\begin{comment}
\begin{figure}[t!]
\begin{center}
\includegraphics[width=0.6\linewidth]{MIscheme.png}
\end{center}
\caption{\label{MIscheme}Schematic of the trapping potentials for (a) single injection and decay and (b) multiple injections before decay.  The single injection trapping scheme uses one cycle of filling the RFQ, injects the bunch into the EBIT, and closes the outermost trap electrode for storage and decay spectroscopy.  The multiple-injection scheme uses successive 25~ms extractions from the RFQ while the charge-bred ions in the EBIT experience a ``deeper" effective potential, and are not lost upon each injection.  After the space-charge limit of the EBIT is reached, the outermost electrode is raised for storage and decay spectroscopy~\cite{Kla15,Lea15b}.}
\end{figure}
\end{comment}

For decay spectroscopy experiments~\cite{Len14,Lea15b,Lea15c,Lea17}, the EBIT is operated with cycles that consists of three parts: injection, breeding/decay, and extraction.  The specific cycling required for the proposed work has been optimized using aforementioned charge-state simulations combined with GEANT4.  To maximize the NEEC signal in this experiment, the EBIT cycling routine will be as follows:
\begin{enumerate}
    \item The purified $^{129m}$Sb$^+$ ions are extracted from the MR-ToF (running at 100 Hz, but accumulating in the re-trapping RFQ) and injected in a single bunch to the EBIT.
    \vspace{-8pt}
    \item The ions in the EBIT are immediately bombarded by a 6.6(1)~keV e-beam for 0.69~s to maximize the number of $^{129m}$Sb$^{41+}$ ions.
    \vspace{-8pt}
    \item The e-gun potential is then quickly ramped to 7.0(1)~keV to stimulate the resonant NEEC process in the $^{129m}$Sb$^{41+}$ ions for 60~s.  The NEEC stimulation $\gamma$ signals are observed during this stage.
    \vspace{-8pt}
    \item The ions are pulsed out of the EBIT after a total trapping time of 60.69~s to a detector downstream to monitor the final charge-state distribution, and the cycle begins again.
\end{enumerate}

\subsubsection{\label{cycling}NEEC Stimulation and Observation}
\begin{figure}[t!]
\begin{center}
\includegraphics[width=\linewidth]{129mSbNEEC_G4Spectrum+bg.png}
\end{center}
  \caption{\small GEANT4 simulated $\gamma$-ray spectrum for NEEC population of $^{129m2}$Sb under the optimized running conditions described here for 7 days of ISAC RIB. An estimate for the room background has also been included, and is the dominant background in our region of interest.  The inset shows signals of the characteristic 1161~keV $\gamma$ line for various NEEC cross section scaling factors alongside the 1128~keV $\gamma$ ray from the decay of $^{128m}$Sb for comparison.  The signal that would result from a NEEC probability 9 orders of magnitude larger than our theoretical estimate is not shown for clarity.}
  \label{SimulatedSpectrum}
\end{figure}
The NEEC stimulation rate is dependent on the EBIT operating conditions, and estimated as
\begin{equation}
    R=\int dV n_i n_e \langle\sigma v\rangle \frac{\Gamma_n}{\Gamma_e},
\end{equation}
where $dV$ is the interaction volume (0.0027~cm$^3$), $n_i$ = $7\times10^{7}$ cm$^{-3}$ and $n_e = 5.6\times10^{11}$ cm$^{-3}$ represent the ion and electron densities, respectively, $\sigma=7\times10^{-6}$ (b eV) is the calculated NEEC cross section for $3p$ capture into $^{129m}$Sb$^{41+}$, $v=5\times10^9$ (cm/s) is the relative velocity of the ions with respect to the electrons, and the ratio $\left(\frac{\Gamma_n}{\Gamma_e}\right)$ gives the extent to which the energy resonance condition is fulfilled.  This ratio term, in particular, is the hardest to estimate since the resonance width for electronic capture into the $3p$ orbit of Ne-like Sb is poorly constrained.  However, for such electron recombination into an excited electronic configuration, the width of the Lorentz profile is determined by the electronic width, typically on the order of $\Gamma_n=1$~eV~\cite{Pal19}.  When combined with $\Gamma_e=100$~eV of our electron beam, this ratio is of order 0.01.    Although the wide electron energy distribution of our current $e$ gun decreases the probability of stimulating NEEC within the capture resonance width, it is somewhat advantageous here since the nuclear and atomic energy level uncertainties are of the same order, and thus allows for less strict tuning requirements for the energy matching.

This gives a NEEC stimulation rate under the above assumptions of $R = 4.2\times10^{-5}$ $s^{-1}$, and when folded with the $\sim40\%$ charge-state population in the EBIT, yields a total of 10 stimulated NEEC events in the trap over 7 days of running.  Although this is roughly four orders of magnitude below our current level of sensitivity, if the NEEC cross section is indeed 9 orders of magnitude larger than the $\sigma$ value given above (as suggested in Ref.~\cite{Chi18}), we will observe a signal within 5 minutes of running.  Using the theoretically estimated cross section, the expected $\gamma$-ray spectrum for 7-days of running (including backgrounds), is shown in Fig.~\ref{SimulatedSpectrum} with various scenarios of the NEEC cross section.  From these estimates, the proposed experiment is sensitive to whether NEEC was indeed observed in the previous experiment~\cite{Chi18} - even under far more pessimistic scenarios for the experimental assumptions included here.  \textbf{If the NEEC stimulation is not observed, we will be able to set a limit on the cross section that is roughly 5 orders of magnitude smaller than the only claimed observation of this process.}

%For the studies outlined in this proposal, this background reduction is crucial, due to the low decay probabilities of the isotopes of interest, and the effective lengthening of these times as the atomic charge-state increases.  From previous in-trap decay measurements with the Si(Li) detectors~\cite{Len14,Lea14b}, this background is roughly 0.04 cps/keV at 110 keV while the trap is full, and more than an order of magnitude lower when the trap is empty. This allows for the detection of weak decay branches via photon counting. Therefore the decay-rate differences between the various charge states can be determined by comparing the relative photo peak intensities listed in section \ref{beamreq}.

\begin{comment}
\subsection{\label{GSI}Note of Complimentary Efforts at the GSI-CRYRING}
%%ID 17/03/21 Text suggestion for GSI-CRYRING exp
We also note here that the GSI-CRYRING collaboration has interest in such a measurement as well.  In fact, the case of $^{129}$Sb was first identified by this group as part of an effort to initiate a more focused NEEC-program at the CRYRING~\cite{Let16}. Since the experimental methods of production, separation, storage, and NEEC stimulation are dramatically different than what we propose here, this effort would be highly complementary and will serve as an independent confirmation of our measurement or limit.  This CRYRING proposal has not yet been submitted to the GSI PAC, but is in preparation.  Even with the promising CRYRING work on the horizon, we see several advantages to our experimental method that we note here.  First, the low-energy environment of the ion trap significantly limits energetic non-resonant processes from exciting the nuclear state outside of NEEC.  Second, the deceleration processes in both storage rings leads to an increased loss of highly-charged ions due to recombination processes, making the NEEC conditions more difficult to satisfy, and for the NEEC events that are generated during implantation, the immediate neutralization of the ions will generate large rates of $\gamma$- and X-rays from non-NEEC decays, making a signal difficult to observe.
\end{comment}

\subsection{Summary}
The process of NEEC has yet to be conclusively shown to exist, and has tremendous implications for both fundamental and applied nuclear physics.  This proposal aims to resolve a 9 order of magnitude discrepancy in the stimulation probability between the only reported observation of NEEC~\cite{Chi18} and the theoretical prediction~\cite{PalPC} by testing the underlying NEEC theory in $^{129m}$Sb.  We will achieve this by performing the NEEC stimulation at energies below the nuclear energy difference (9.8~keV) in an EBIT to prevent any non-resonant processes from exciting the state of interest.  The controlled environment of the EBIT will allow us to selectively tune the NEEC condition which will be characteristically identified by the correlated observation of a $\gamma$ line at 1161~keV.  The clean region of signal, and the control of the EBIT environment will allow us to set the first limits on the NEEC probability, even if it is not observed.


%The $^{129}$Sb$^m$ will be produced by projectile fission in the Fragment Separator as fully ionized or few-electron system and the (quasi) mono-isomeric beam injected in the storage ring ESR. In the ESR the beam will be slowed down to a few MeV/u and can be further purified before injecting it into the CRYRING where it will be decelerated to a few tens of keV/u and then implanted on a Moving Tape Collector where the characteristic $\gamma$-rays of the NEEC de-excitation process are detected.

\section{Experimental Equipment}
\subsection{TITAN-Specific Equipment}
The performance of the three ion traps that are required for these measurements (RFQ, MR-ToF, and EBIT) have each been demonstrated with RIB on multiple occasions and are ready to accept beam for these measurements when scheduled.  The decay spectroscopy setup with the EBIT has also been used for several measurements over the past decade, and has been upgraded recently to accommodate the required HPGe detectors.

\subsection{TRIUMF Contributed Equipment}
The primary support we need from TRIUMF-ISAC is the use of IG-LIS to provide selective ionization of antimony, with a suppression of the surface-ionized cesium.  We also request that beam is sent to the yield station prior to the experiment to determine the ratio of $A=129$ species in the beam.

\section{Readiness}
The TITAN facility has successfully demonstrated in-trap decay spectroscopy on HCIs in previous measurements~\cite{Len14,Lea15b}.  For the installation of the HPGe detectors, the design work is complete, and fabrication of the support components is currently underway in the TRIUMF machine shop, and will be completed in April.  Since these detectors have much smaller LN2 dewars, an autofill system is partially completed and should be ready for the summer.  If this is delayed, we can use a stationary manual fill dewar connected to the system.  The existing HV supplies and DAQ infrastructure have been recently upgraded and tested.  The infrastructure on the EBIT side already exists to perform the measurement, and the NEEC rate estimates can be significantly improved by installing and testing the new EBIT electron gun, but is not required.  In short, the TITAN setup is ready to perform this experiment once it is scheduled, with roughly 4 weeks of setup and optimization preparation time.

\section{Beam Estimates and Shift Requests}
Our rate estimates for ISAC production of $^{129m1}$Sb are taken from the previously measured data shown in Fig.~\ref{129spectrum} (left).  This spectrum was acquired in the TITAN MR-ToF using beam from a 10~$\mu$A proton current on a UCx+IGLIS target (lasers tuned to In) and a $10\times$ attenuator in the ISAC beamline, and shows roughly 10 pps of $^{129m}$Sb.  Assuming the canonical suppression factor for surface-ionized species of the IGLIS of $10^6$, it suggest that $10^7$ pps of surface-ionized $^{129m}$Sb were blocked by the repeller plates.  Although it is likely that the $^{129m}$Sb rate will exceed this when the proton current is increased and lasers are selectively tuned to antimony, here we assume an ISAC rate delivered to TITAN of $1\times10^7$ $^{129m1}$Sb s$^{-1}$ with a beam composition of Cs:Sb(m1):Sb(gs) = 10:4:1.

\begin{table}[t!]
\caption{\label{tab:beamtime}Proposed activities and beam request for each - presented in 8-hour shifts.  The measurement time requested is based on the stimulation of 10 NEEC events under the experimental assumptions for the ISAC beam rates, trapping efficiencies, and NEEC cross-sections described in this proposal.}
\begin{center}
\begin{tabular}{cccc}
\hline
\hline
Isotope & Facility & Description & Shifts Required\\
\hline
OLIS Pilot Beam (Cs) & TITAN & Setup and Tuning & 2 \\
$^{129m}$Sb & Yield Station & $A=129$ Beam Rate/Content & 1 \\
$^{129m}$Sb & TITAN & NEEC measurement (Ne-like) & 21 \\
\hline
 & & \hfill{\bf Total} & {\bf 22 RIB } (2 SIB)\\
\end{tabular}
\end{center}
\end{table}

The shift request for this experiment is shown in Table~\ref{tab:beamtime}.  The ISAC rate of $10^7$ $^{129m}$Sb$^+$ pps generates a pure sample of roughly $2\times10^5$ $^{129m}$Sb$^+$ ions per cycle (every 60~s) from the MR-ToF into the EBIT.  Using these estimates along with the EBIT conditions, optimal cycling, and the theoretically predicted NEEC stimulation cross-section, we anticipate stimulating 10 NEEC events in 7 days of running.  These estimates can be significantly improved if the new EBIT e-gun (currently at TRIUMF and ready for installation) is installed in time for testing prior to the experiment, however this is not required to achieve the primary goal of this experiment.

\section{Data Analysis}
A full suite of data-analysis programs have been developed and thoroughly tested by the two PhD students who will be responsible for the analysis.  The data acquired will be used for the theses of these two senior PhD students: Jon Ringuette (Mines) and Zach Hockenbery (McGill).  The data will be stored both locally at TRIUMF and on the data server at the Colorado School of Mines.

%\clearpage

\section{References}
%\nocite{apsrev41control}
\bibliography{references}
\end{document}
