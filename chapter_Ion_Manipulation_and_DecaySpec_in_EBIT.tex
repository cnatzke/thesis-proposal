\documentclass[jon_ringuette_thesis_proposal.tex]{subfiles}
\begin{document}

    \chapter{Ion Manipulation and Decay in an EBIT}
    In order to measure NEEC a number of different systems are needed.
    The overall idea is to use an Electron Beam Ion Trap (EBIT) to both trap and further ionize an ion bunch.
    Once stored, the electron beam energy can be rapidly changed to meet the NEEC resonant condition.
    Photon detectors will then be used to try to observe the decay from the exited state generated from the triggering of NEEC.

    \subsection{TITAN's Electron Beam Ion Trap}
    The Electron Beam Ion Trap (EBIT) is a piece of equipment containing both an ion trap and an electron gun.
    \begin{figure}[H]
        \begin{center}
            \includegraphics[scale=.4]{ebit_w_detectors.png}
        \end{center}
        \caption{EBIT on TITAN platform in current configuration with detectors plumbed and wired.}
        \label{fig:ebit_w_detectors}
    \end{figure}

    \begin{figure}[H]
        \begin{center}
            \includegraphics[scale=.55]{ebit_jons_attempt_annotated.png}
        \end{center}
        \caption{\small EBIT overview : Pink squares represent superconducting magnets and the field lines generated. On the right is the electron gun with the corresponding electron compression path through the elongated green drift tubes. $\gamma$-rays that pass-through Be windows surrounding the central trapping region where the HPGe detectors sit.}
        \label{fig:ebit_field_overview}
    \end{figure}

    For trapping in the EBIT, the trapped radioactive ions are axially confined in an electrostatic potential generated by the cold drift-tube assembly.
    Further, they are radially confined by the strong magnetic field, this field is supplied by superconducting magnets composed of Nb3Sn and cooled to $\approx4.5K$ providing a field of up to 6T.
    Additional confinement is provided by the space-charge potential of the electron beam \cite{TITAN_EBIT_2010}.
    When trapped, the ions are under constant electron bombardment thus increasing their charge state through electron-impact ionization.
    The resulting equilibrium charge-state distribution is defined by several parameters including the electron-beam current, density, potential, as well as the trapping time, the vacuum in the trap, and decay lifetime.
    The current electron gun installed on the EBIT is capable of generating 500mA of electron current.

    \begin{figure}[H]
        \begin{center}
            \includegraphics[scale=.3]{EBIT1}
        \end{center}
        \caption{\small Internal cutout view of the EBIT. \cite{Leach}}
        \label{fig:ebit_schematic_1}
    \end{figure}


    The average charge breeding time needed by the EBIT to charge breed to a specific charge state from a neutral atom is given by Eq. \ref{eq:ebit_charge_breed_time} \cite{SPRINGER_1997}.

    \begin{equation}
        \tau_i(q) = \sum^{q-1}_{q'=0}\langle n_e v_e \sigma^I_{q', q'+1} \rangle^{-1}
        \label{eq:ebit_charge_breed_time}
    \end{equation}

    Where $\tau_i$ is the charge breeding time, $q$ is the charge state, $n_e$ is the electron density, and the single ionization cross-section is $\sigma^I_{q', q'+1}$.

    The EBIT uses electron impact to ionize the target ions therefore it is important to keep in mind that there will be a charge distribution in the EBIT however threshold charge-breeding can be applied as long as there is a reasonable gap between the ionization levels.
    This also implies that this technique is best used when breeding to closed shells such as K, L, or M shells.
    \begin{figure}[H]
        \begin{center}
            \includegraphics[scale=.4]{egun_in_chamber}
        \end{center}
        \caption{EBIT electron gun chamber with electron gun shown prior to removal for cathode replacement.}
        \label{fig:egun_in_chamber}
    \end{figure}

    \subsubsection{EBIT for Decay Spectroscopy}

    The unique open-geometry of the TITAN EBIT \cite{TITAN2015} allows for direct access to the ion trapping region via 7 access points laid out perpendicular to the drift tubes.
    These 7 ports are spaced $45^{\circ}$ apart, each with an opening of 35mm in radius with optional Be windows to ensure high vacuum in the trap center $\sim 10^{-9} Torr$ See \ref{fig:ebit_geometric_acceptance}.

    \begin{figure}[H]
        \begin{center}
            \includegraphics[scale=.3]{EBIT1_HPGe}
        \end{center}
        \caption{\small Cross-sectional view of the EBIT with 7 acceptance ports \cite{TITAN2015}}
        \label{fig:ebit_geometric_acceptance}
    \end{figure}

    The Be windows function to keep vacuum and also contain any $\alpha$ or $\beta$ radiation from escaping while allowing $\gamma$-rays to easily pass through even when at very low energies of a few keV.
    The placement of the Be windows allow for detectors to be placed within (226 mm or 230 mm) away from the trap center.
    When all 7 ports are occupied by detectors then the geometric acceptance can be up to $2\%$ of $4\pi$.

    %%%%%%%%%%%%%%%%%%%%%%%%%%%%%%%%%%%%%%%%%%%%%%%%%%%%%%%

    \subsection{$\gamma$-ray detection using High Purity Germanium Detectors}
    In order to do decay spectroscopy on the EBIT it was decided that several High Purity Germanium Detectors (HPGe's) would be utilized due to their wide energy sensitivity and general operational stability.

    For $\gamma$-ray spectroscopy detectors require a large depleted volume in order to allow the incoming $\gamma$-ray to impart all of its energy into the volume.
    As of this writing one of the best ways to obtain a large depleted volume is via high purity germanium crystals.
    The depth of a depletion volume can be calculated from (Eq. \ref{eq:depletion_depth}) where d is the thickness or depth of the depleted region, V is the reverse bias voltage, N the net impurity concentration in the bulk semiconductor material, $\epsilon$ the dielectric constant and $e$ the electronic charge \cite{KNOLL}.

    \begin{equation}
        d = \left( \frac{2 \epsilon V}{e N} \right)^{1/2}
        \label{eq:depletion_depth}
    \end{equation}

    From this equation we observe that to create a greater depletion depth one need only increase the voltage and reduce the impurity in the semiconductor material.
    While increasing the voltage to the breakdown voltage is easy enough reducing the impurities is not.
    To reduce the impurities in the commonly available semiconductors (Silicon, and Germanium) a process of melting local regions of the sample whereby impurities are preferentially transferred to the molten region and removed from the sample \cite{KNOLL}.
    Due to the differing melting points in common high purity semiconductors, $1410 ^{\circ} C$ for silicon and $959 ^{\circ} C$ for germanium the choice of using germanium becomes obvious.
    Once this processing of melting and re-melting has removed a sufficient proportion of impurities on the order of $10^9 atoms/cm^3$ the sample is then used as feedstock for growing the ultra-high purity germanium crystal used in the High Purity Germanium detectors (HPGe).

    These detectors are operated at around $77K$ via the application of liquid nitrogen (LN2) to a cold finger located near the germanium.
    This is due to the small band-gap inherent in the material ($0.7 eV$).
    This band-gap causes any operation at room-temperature to be infeasible due to the resulting large thermally-induced leakage current \cite{KNOLL}.

    \subsubsection{Coaxial HPGe - Midrange 100keV to 2MeV}

    \begin{figure}[H]
        \begin{center}
            \includegraphics[scale=.1]{8pi_in_box.pdf}
        \end{center}
        \caption{\small HPGe, former 8pi detector made by Ortec}
        \label{fig:8PI_BOX}
    \end{figure}

    The NEEC experimental setup consists of employ 5 Ortec HPGe's (\ref{fig:8PI_BOX}) formally used in the 8$\pi$ experiment for the detection of medium energy $\gamma$-rays.
    In order to detect medium energy $\gamma$-rays a large depletion area is required for the $\gamma$-ray to deposit all of it's energy into the Ge crystal.
    To accomplish this a coaxial design (\ref{fig:NTYPE_COAXIAL}) is used.
    As is common, these detectors have a small diameter inner radius that is introduced in order to lower capacitance as well as keeping the field lines as radial as is possible \cite{KNOLL}.
    For calculating the breakdown voltage needed ($V_d$) to fully deplete the crystal causing the semiconductor Ge to become conductive Eq. \ref{eq:coaxial_depletion_depth} is used.

    \begin{equation}
        V_d = \frac{\rho}{2\epsilon} \left[ r^2_1 ln\left(\frac{r_2}{r_1}\right) - \frac{1}{2}(r^2_2-r^2_1)\right]
        \label{eq:coaxial_depletion_depth}
    \end{equation}

    Here $r_1$ and $r_2$ are the inner and outer radii of the coaxial crystal respectively and $\rho$ is the charge density.

    \begin{figure}[H]
        \begin{center}
            \includegraphics[scale=.25]{ntypecoaxial.png}
        \end{center}
        \caption{\small Left: Cross sectional view through axis of large volume bulletized, closed ended, coaxial crystal. Right: Perpendicular cross sectional view of n-type high purity coaxial germanium crystal. Inspired by \cite{KNOLL}}
        \label{fig:NTYPE_COAXIAL}
    \end{figure}

    As can be seen in \ref{fig:NTYPE_COAXIAL} for a n-type coaxial detector, of which the Ortec's used are, the electric field causes the electrons to want to flow towards the inner radius and the holes to flow in the opposite direction allowing for the current generated by the traveling electrons which were freed due to the interaction of the $\gamma$-ray with the Ge material to be easily detected.

    % *** PUT IN SOME EFFICIENCY GRAPHS ONCE WE HAVE THEM AND CALIBRATION GRAPHS ***

    \subsubsection{Planer Ultra Low Energy HPGe - 1keV to 200keV}
    In order to observe x-rays given off by trapped ions in the EBIT an Ultra Low Energy Germanium (ULGe) detector is used \ref{fig:ULGE_NEW}.
    Due to its primary use in observing x-rays to determine the charge state within the trap this detector is operated within the EBIT's vacuum and does not have a Be window between it and the trapping region.

    \begin{figure}[H]
        \begin{center}
            \includegraphics[scale=.1]{ulge_new.png}
        \end{center}
        \caption{\small Ultra Low Energy HPGe made by Canberra}
        \label{fig:ULGE_NEW}
    \end{figure}

    This type of detector has a planar configuration to it's high purity Ge crystal (\ref{fig:ULGE_planar_config}).
    The planar type of configuration allows higher energy $\gamma$'s through while still being sensitive to x-rays in the single digit keV range.
    This type of detector is optimal for determining the charge state of ions trapped in the EBIT due to the electron stripping causing the emission of x-rays.
    \begin{figure}[H]
        \begin{center}
            \includegraphics[scale=.35]{planarconfig_1.png}
        \end{center}
        \caption{\small Planar configuration. Inspired by \cite{KNOLL}}
        \label{fig:ULGE_planar_config}
    \end{figure}

    Akin to the HPGe's the ULGe's voltage breakdown depth can be calculated via Eq. \ref{eq:planar_depletion_depth}
    \begin{equation}
        V_d = \frac{\rho T^2}{2 \epsilon}
        \label{eq:planar_depletion_depth}
    \end{equation}

    Where $\rho$ again is the charge density.

    For the selected ULGe an efficiency measurement is given by Canberra \ref{fig:ULGE_EFFICIENCY} indicating that it's region of optimal energy detection is from 0keV - 100keV.
    \begin{figure}[H]
        \begin{center}
            \includegraphics[scale=.4]{ulge_efficiency.png}
        \end{center}
        \caption{\small ULGe detection efficiency. \cite{mesytec_mdpp16_manual}}
        \label{fig:ULGE_EFFICIENCY}
    \end{figure}

    %%%%%%%%%%%%

    \subsubsection{High and Low voltage power supplies}
    The HPGe detectors all require a high negative bias voltage on the order of $-2000V$ to $-3000V$. As this voltage directly affects electron transport through the Ge crystal it is critical to have a supply a clean consistent bias voltage to each HPGe.
    Small voltage fluctuations during the experiment can cause changes in the bin to energy calibrations and thus distort $\gamma$-ray peaks.
    The CAEN R8034N is a compact rack mounted 8 channel HVPS that suites the experimental needs well.

    \begin{figure}[H]
        \begin{center}
            \includegraphics[scale=.12]{HVPS_w_DAQ.png}
        \end{center}
        \caption{\small CAEN model R8034N HVPS for HPGe's w/ VME DAQ}
        \label{fig:CAEN_HVPS}
    \end{figure}

    Each of the units eight channels is independently controllable and has a voltage range of 0 to $-6000V$.
    The R8034N is also programmable and able to set voltage ramping rate limits independently per channel to ensure a Ge crystal is not damaged by too quickly biasing or un-biasing.
    To further protect the HPGe detectors the R8034N can use the detectors INHIBIT signal both using TTL ($\ge +2V$) and ORTEC low voltage ($-24V$) modes.
    The detector, in event of a crystal starting to warm up, will trigger an INHIBIT signal to the R8034N which will then ramp down the voltage for that channel allowing the HPGe to safely warm up without damaging the delicate Ge crystal.
    This is a network enabled unit and can be controlled remotely via a web interface or C-API.
    To control the HVPS the C-API interface was used.
    This was then wrapped in Python code allowing for the construction of a command line interface \cite{pyhvps_software} to adjust voltages and other parameters.

    \begin{figure}[H]
        \begin{center}
            \includegraphics[scale=.3]{preamp_N5424.png}
        \end{center}
        \caption{\small CAEN model N5424 preamp low voltage power supply. \cite{CAEN_N5424}}
        \label{fig:CAEN_PREAMP_PS}
    \end{figure}

    To supply power to each of the HPGe detector's pre-amplifiers two N5424 NIM based pre-amp power supplies were used \ref{fig:CAEN_PREAMP_PS}.
    Each of the N5424's requires a NIM compatible slot and can supply up to 4 pre-amplifiers with voltages of $\pm 6V$, $\pm 12V$, and $\pm 24V$ \cite{CAEN_N5424}.
    Each voltage is internally surveyed to ensure accurate voltage output with a red or green LED showing the status of each voltage output per channel.
    These units also filter each power line separately to create a low level of voltage noise \cite{CAEN_N5424}.

    Both the high and low voltage power supplies along with the entire DAQ system are protected by a Tripp-Lite SU1000RTXLCD2U UPS battery backup unit.
    This unit provides power conditioning with a 1kVA on-line, pure sine wave capability with an output voltage stability of $\pm2\%$ \cite{Lennarz2015}.
    This unit also provides time to properly unbias the detectors in the event of a facility wide power outage or brownout.

    %  Spend ~10 or so pages talking about the DAQ and sorting code

    % \subsubsection{GRIFF16 Analog to Digital Converter}

    %\subsubsection{MDPP16 Analog to Digital Converter}

    \subsubsection{MIDAS (Maximum Integrated Data Acquisition System) \cite{Garnsworthy2017a} }

    The MIDAS software is employed at TRIUMF to manage a myriad of incoming raw instrument data and provides a framework to save, decode, and perform basic online analysis for said data.
    As shown in \ref{fig:ADC_TO_MIDAS} the MIDAS system has several layers of interacting programs which are largely written in C/C++.

    \begin{figure}[H]
        \includegraphics[scale=.38]{adctomidas.png}
        \caption{\small Diagram of data flow from ADC's to output through MIDAS}
        \label{fig:ADC_TO_MIDAS}
    \end{figure}

    Starting from the left on \ref{fig:ADC_TO_MIDAS} raw Analog to Digital Converter (ADC) data from both the GRIF16, used for the HPGe detectors, and MDPP16, which is used for the ULGe detector, is sent to a custom written frontend programs.
    These frontend programs provide the functionality of initializing the ADCs, setting any required attributes in the ADCs as well as determine the beginning and end of ADC's data packets to pass onto MIDAS.
    In the case of the MDPP16 which the signal is passed via a Versa Module Eurocard (VME) backend data bus rather than TCP and thus requires the use of MSERVER.
    MSERVER is used when a data frontend is on a remote system and not located on the same computer as the MIDAS software.
    Once the data packets arrive at primary MIDAS application the data is then sent to both the MLOGGER application as well as the Analyzer application.
    The MLOGGER application is responsible for wrapping the raw data packet from the ADC in a MIDAS data packet which contains various additional pieces of metadata.
    The Analyzer component is responsible for decoding the raw binary data packet for use by the online analyzer.
    In this case that is hit value and detector channel number allowing for building of basic histograms in near realtime.

    \subsubsection{Event Decoding}
    Both the GRIF16 and MDPP16 output raw binary data to MIDAS and thus require decoding at many stages.
    This is handled via binary bit shifts guided by \ref{fig:GRIF16_GRF4} for the GRIF16 and by Appendix \ref{sec:MDPP16} for the MDPP16.
    For both ADCs a data packet consists of a number of 4 byte words used to separate out different parts of the data packet such as header, footer, (Time to Digital Converter) TDC, and ADC values.

    \begin{figure}[H]
        \includegraphics[scale=.35]{grif16_GRF4_format.png}
        \caption{\small GRIF16 - GRF4 Data Format \cite{GRIF16_format_November2015}}
        \label{fig:GRIF16_GRF4}
    \end{figure}


    To decode these data packets a packet is read into memory and divided into an array of 4 byte words.
    Each word contains a few bits at the beginning which indicate what type of data the word represents.
    For instance the GRIF16 uses the first 4 bits of the word, those 4 bits must be converted to hex and then compared to the "Packet Type" column in \ref{fig:GRIF16_GRF4}.
    If it is deemed that the "Packet Type" found is relevant then the next number of bits is decoded in a similar fashion following the packet format.

    For the GRIF16 and MDPP16 the MIDAS frontend decodes only enough information to determine that there was a detected pulse and how many pulses.
    Whereas the MIDAS Analyzer component as well as the sorting code do a more thorough job of decoding these data packets.

    To demonstrate the complete DAQ setup and offline decoding a number of spectra were taken both for testing and for calibration.  Below in \ref{fig:hpge_calibration_co60} is a full demonstration of the stack using a single HPGe detector hooked up to the EBIT.  The $^{60}Co$ source was placed behind the HPGe detector head and then decoded and loaded into the graphing software package that was designed for this experimental setup.
    \begin{figure}[H]
        \centering
        \begin{mdframed}[backgroundcolor=black!80]
            \includegraphics[scale=0.12]{8pi_co60_cal}
        \end{mdframed}
        \caption{$^{60}Co$ energy spectrum taken by HPGe detector while attached to the EBIT.}
        \label{fig:hpge_calibration_co60}
    \end{figure}

    In addition, the ULGe was tested in a similar manner with a $^{133}Ba$ source.
    The $^{133}Ba$ source was placed outside the Ba window on the opposite side of the EBIT causing the photons to travel through the EBIT trapping region.
    \begin{figure}[H]
        \begin{center}
            \begin{mdframed}[backgroundcolor=black!80]
                \includegraphics[scale=0.12]{ulge_ba133_cal}
            \end{mdframed}
        \end{center}
        \caption{$^{133}Ba$ energy spectrum taken by ULGe detector while attached to the EBIT.}
        \label{fig:ulge_calibration_ba133}
    \end{figure}

    \subsubsection{GammaGraph - Online and Offline Histograms}

    GammaGraph was designed to facilitate equipment diagnostics, offline data analysis and collaborative online, real-time data analysis.
    Using the Python Dash API GammaGraph was built to function via a web based interface to allow multiple users to collaborate.
    As seen in \ref{fig:gammagraph_linear} this software is capable of loading multiple offline histogram files simultaneously allowing for the comparison of spectra.

    \begin{figure}[H]
      \begin{center}
        \includegraphics[scale=.3]{gamma_graph_linear.png}
      \end{center}
      \caption{GammaGraph software comparing two spectra in offline mode.}
      \label{fig:gammagraph_linear}
    \end{figure}

    Viewing options in the software allow for Log and Linear views as well as peak fitting to determine the Full Width Half Maximum  (FWHM) on the selected peak.
    The peak information is saved on the server running GammaGraph so that any other collaborators who load the spectra will see the fitted peaks along with the peaks FWHM measurement.
    This is best shown in \ref{fig:hpge_calibration_co60} and \ref{fig:ulge_calibration_ba133} were all the labeled peaks were fitted and automatically labeled within the software.
    Those energy spectra were calibrated from pulse height to their corresponding energy values by uploading the calibration file generated via the auto-calibration software written for this experiment into GammaGraph.

    In addition to basic viewing and fitting of offline histograms the software has an online mode.
    When functioning in the online mode it is capable of showing spectra in real time from any of the detectors individually or all simultaneously.
    In order to facilitate charge breeding optimizations the online view mode can produce a heatmap where the y-axis is time, the x-axis is pulse height and the color represents the counts at that pulse height.
    This facilitates determining when a certain charge state is reached by examining when characteristic x-rays are observed in time indicating the desired charge state.


%%%%%%%%%%%%%%%%%%%%%%%%%%%%%%%%%%%%%%%%%


\end{document}