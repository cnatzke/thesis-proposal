%! Author = cnatzke
%! Date = 05/24/2022
\documentclass[cnatzke_thesis_proposal.tex]{subfiles}
\usepackage{amssymb}
\begin{document}

%------------------------------------------
\subsection{Nuclear Equation of State}
%------------------------------------------
The Nuclear Equation of State (EOS) governs the properties of nuclear matter and defines the properties of neutron-rich nuclei, the structure of neutron stars, supernova explosions and compact object mergers \cite{knupfer_scaling_1985}. 
The EOS is well constrained for symmetric nuclear matter, matter where the number of protons and neutrons are roughly equal, but the properties for asymmetric, neutron rich matter require further investigation \cite{danielewicz_determination_2002}.
The largest uncertainties in the EOS for neutron rich matter lie in the limited knowledge of the symmetry energy $J$, the difference between the energies of neutron and nuclear matter at saturation density, and the slope of the symmetry energy $L$, which relates the pressure of neutron matter. 


Atomic nuclei provide a terrestrial probe of the symmetry energy $J$, where it contributes to the formation of neutron skins in systems with a neutron excess. 
$J$ and $L$ can be correlated with isovector collective excitations of nucleus, such as the pygmy dipole resonance \cite{carbone_constraints_2010} and giant dipole resonances (GDRs) \cite{trippa_giant_2008}, via calulations based on energy density functionals (EDfs) suggesting the neutron skin thickness, the difference between the point-neutron and point-proton radii $R_{skin} = R_n - R_p$ \cite{tsang_constraints_2012}, could be constrained by measuring isovector collective observables at low energy \cite{krasznahorkay_excitation_1999}.
A direct experimental measurement of $R_{skin}$ is extremely challenging and can be extracted by its correlation to collective isovector observables at low energies \cite{birkhan_electric_2017}. 
One such observable is the dipole polarizability $\alpha_D$ \cite{birkhan_electric_2017}, representing an experimentally viable tool to constrain the EOS of neutron matter and the physics of neutron stars. 
% TODO : Add citations to last sentence. 6-11 from birkham

%------------------------------------------
\subsection{Electric Polarizability and Magnetic Susceptibility}
%------------------------------------------
The concept of polarizability, a fundamental concept in physics the describes how applied electric or magnetic fields induce an electric or magnetic multipole moment in the matter of interest \cite{jackson_classical_1999}, plays a large role in nuclear physics observables across a broad range of interests.
Of specific interest to the EOS is the electric and magnetic dipole polarizability of the nucleus because nuclear DFT calculations suggest a strong correlation between $\alpha_D$ and the neutron radius \cite{hagen_neutron_2016}, $R_{n}$. 
The static dipole polarization of the ground and excited state shape of atomic nuclei is influenced by the state's coupling to higher-energy collective modes, like the giant dipole resonance and pygmy dipole resonance, via virtual excitations. 
As described in \cite{soderstrom_electromagnetic_2020}, the static dipole polarizability, $\alpha_D$ can be obtained from the photonuclear population of excited states, given by

\begin{equation} \label{eqn:scalar_dipole_polarizability}
    \alpha_{D,E1} = 2 e \sum_n \frac{|\langle I_0 || E1 || I_n \rangle |^2}{E_n - E_0}
\end{equation}

where $I_0$, $E1$, and $I_n$ are the matrix elements for the ground state, electric dipole transition, and excited state respectively; $e$ is the elementary charge of the electron; and $E_n$ is the energy of the state. 
The value of $alpha_D$, and the analogous magnetic susceptibility $\chi$, in atomic and nuclear systems spans many orders of magnitude depending on the scale of interest, for example the magnetic susceptibility of a nucleus is twelve orders of magnitude smaller than that of an atom \cite{knupfer_scaling_1985}.

Following the explanation in \cite{soderstrom_electromagnetic_2020} the polarizability can be expanded past the scalar case to include separate tensor components describing higher-order electric and magnetic multipoles. 
Using the nuclear structure framework these off diagonal polarizabilities appear as very weak second-order electromagnetic processes that can be expressed analogously to Eq. (\ref{eqn:scalar_dipole_polarizability}) in terms of either electric and magnetic components or different multipolarities. 

\begin{equation} \label{eqn:m2e2_dipole_polarizability}
    \alpha_{M2E2} = \sum_n \frac{\langle I_f || E2 || I_n \rangle \langle I_n || M2 || I_i \rangle}{E_n - \omega}
\end{equation}

or 

\begin{equation} \label{eqn:m2e2_dipole_polarizability}
    \alpha_{E3M1} = \sum_n \frac{\langle I_f || M1 || I_n \rangle \langle I_n || E3 || I_i \rangle}{E_n - \omega}
\end{equation}

where $\omega$ is the interference frequency of the emitted gamma-rays and is approximated to one half of the initial energy. 
This second-order electromagnetic process was discussed in the doctorate dissertation of Maria G\"oppert-Mayer~\cite{goppert-mayer_uber_1931} where the estimated relative probability of two-photon absorption to the single-photon process is $10^{-7}$. 

The electric dipole polarizability has been measured for a variety of stable isotopes, $^{48}$Ca \cite{birkhan_electric_2017}, $^{120}$Sn \cite{hashimoto_dipole_2015}, and $^{208}$Pb \cite{tamii_complete_2011} using proton inelastic scattering, and $\alpha_D$ has only been measured for one short-lived nucleus, $^{68}$Ni~\cite{rossi_measurement_2013}.
At this time the electric dipole polarizability for a short-lived nuclear state has not been experimentally measured, but using the \textit{transition} electric polarizabilty, the difference in $\alpha_D$ between the two states, and the ground state polarizabilty the polarizabilty of a short-lived nuclear state can be measured.

%%%%%%%%%%%%%%%%%%%%
\subsection{Nuclear Gamma Decay and Internal Conversion}
%%%%%%%%%%%%%%%%%%%%
Nuclear gamma decay is an electromagnetic process in which an unstable nucleus deexcites into a more energetically favourable state via the emission of a photon of gamma radiation. 
The kinetic energy of the photon is equivalent to the energy difference in the initial and final states of the nucleus, save for a small amount of recoil kinetic energy imparted to the nucleus, and the photon's angular momentum equals the difference in angular momentum of the nucleus' state. 
This restriction on angular momentum forbids the emission of a $\gamma$-ray from two states with equal angular momentum, a photon must have at least 1$\hbar$ unit of angular momentum, denoted as an E0 transition.

Gamma-decay often competes with another first-order decay mode called Internal Conversion. 
Internal conversion (IC) is an electroweak process where the nucleus deexcites by transferring its energy to an atomic electron. 
Internal Conversion is particularly important for transitions where both the initial and final states have a $0^+$ spin-parity since IC is the only direct process by which the nucleus can de-excite~\cite{krane_introductory_1987}.

%%%%%%%%%%%%%%%%%%%%
\subsection{Nuclear Two-photon Decay}
%%%%%%%%%%%%%%%%%%%%

The only known experimental probe for the transition electric dipole polarizabilty, $\alpha_D^{fi}$ is nuclear two-photon decay.
Nuclear two-photon decay is a second-order QED process wherein the nucleus simultaneously emits two photons of continuous energy that sum to the initial excitation energy, and the two photons couple to preserve the change in angular momentum of the transition \cite{kramp_nuclear_1987}.
Figure~\ref{fig:two_photon_schematic} graphically demonstrates a non-competitive two-photon transition.

\begin{figure}[H]
    \centering
    \includegraphics[width=0.75\textwidth]{two_photon_schematic.pdf}
    \caption{Schematic representation of two photon decay.}
    \label{fig:two_photon_schematic}
\end{figure}

Nuclear two-photon decay is a fully competitive process, albeit second-order, and can occur between any two states in the nucleus. 
This makes it difficult to observe between levels  for transitions where single gamma decay is allowed. Recently competitive two-photon decay has been measured in $^{137m}$Ba \cite{soderstrom_electromagnetic_2020} but is more easily observed in nuclei that have $0^+$ ground and first excited states since the only allowed decay modes for E0 transitions are internal conversion and two-photon decay. 
This so-called noncompetitive two-photon decay has been observed in the doubly magic nuclei $^{16}$O, $^{40}$Ca \cite{schirmer_double_1984}, and $^{90}$Zr~\cite{schirmer_double_1984}. 

%Nuclear two-photon decay has been observed in several nuclear where the first excited state and ground state have a $J^{\pi} = 0^+$ making single gamma decay is forbidden. 
%In these nuclei the only allowed decay modes between the first excited and ground states are Internal Conversion and two-photon decay. 
%So-called non-competive two-photon decay has been observed in the doubly magic nuclei $^{16}$O, $^{40}$Ca \cite{schirmer_double_1984}, and $^{90}$Zr \cite%{schirmer_double_1984}. 
%In these measurements the energy and angular distribution between emitted photons have been used to investigate the electromagnetic multipole characteristics of two-photon decay and assign a relative branching ratio to the emission of electric to magnetic dipoles. 

Nuclear two-photon decay is expected to be dominated by three matrix elements, the electric transition polarizabilities $\alpha_{E1}$ and $\alpha_{E2}$ and the total transition susceptibility $\chi$, which determine the emission of two E1, two E2, or two M1 photons \cite{kramp_nuclear_1987}.
The differential decay rate ($\hbar = c = 1$) for two-photon decay is given by 
\begin{align} \label{eqn:diff_decay_rate_full}
    \begin{split}
    \frac{d^2\Gamma_{\gamma\gamma}}{d \cos\theta_{12} dE_{\gamma_1}} = \frac{1}{\pi} E_{\gamma_1}^3 E_{\gamma_2}^3 \cdot & \left[ \alpha^2_{E1} + \chi^2 + \frac{1}{6} E_{\gamma_1} E_{\gamma_2} \alpha_{E2} \chi + \frac{E_{\gamma_1}^2 E_{\gamma_2}^2 \alpha_{E2}^2}{(12)^2} + 4 \alpha_{E1} \chi \cos\theta_{12} \right. \\ 
    &\left. + \left(\alpha_{E1}^2 + \chi^2 - \frac{E_{\gamma_1} E_{\gamma_2}}{2} \alpha_{E2} \chi - \frac{3 E_{\gamma_1}^2 E_{\gamma_2}^2 \alpha_{E2}^2}{(12)^2}\right) \cos^2\theta_{12} \right. \\
    &\left. - \frac{E_{\gamma_1} E_{\gamma_2} \alpha_{E1} \alpha_{E2}}{3} \cos^3\theta_{12} + \frac{4 E_{\gamma_1}^2 E_{\gamma_2}^2 \alpha_{E2}^2}{(12)^2} \cos^4\theta_{12} \right]
    \end{split}
\end{align}
and     
\begin{equation} \label{eqn:two_photon_energy_sum}
    E_{\gamma_1} + E_{\gamma_2} = E_0
\end{equation}
where $E_{\gamma_1}$, $E_{\gamma_1}$, and $E_{0}$ are the energy of the two photons and the transition energy; $\alpha_{E1}$, $\alpha_{E2}$, and $\chi$ are the electric transition polarizabilities and susceptibility; and $\cos\theta_{12}$ is the angle between the two photons. 

Equation \ref{eqn:diff_decay_rate_full} leads to the expected energy sharing distribution for a pure dipole transition of 
\begin{equation}
    \frac{d^2\Gamma_{\gamma\gamma}}{d \cos\theta_{12} dE_{\gamma_1}} =  E_{\gamma_1}^3 E_{\gamma_2}^3 \cdot \frac{\alpha_{E1}}{\pi} \left( 1 + \cos^2\theta_{12} \right)
\end{equation}
and for a pure quadrupole
\begin{align}
    \frac{d^2\Gamma_{\gamma\gamma}}{d \cos\theta_{12} dE_{\gamma_1}} = E_{\gamma_1}^5 E_{\gamma_2}^5 \cdot \beta \left( 1 - 3 \cos^2\theta_{12} + 4 \cos^4 \theta_{12} \right) 
\end{align}
where $\beta = \frac{\alpha_{E2}^2}{\pi (12)^2}$.

According the Schirmer \textit{et al.} two-photon decay in $^{90}$Zr and $^{40}$Ca occurs not only through two E1 emission but also through equally strong 2M1 emission leading to a linear interference term in the angular correlation function linear with respect to $\cos\theta_{12}$. 
Kramp \textit{et al.} found the relative branching ratio of 2E2 to 2E1 decays was on the order of $<0.8\%$ with theoretical upper limit of $0.1\%$ meaning the approximmation $\alpha_{E2} = 0$ equation \ref{eqn:diff_decay_rate_full} becomes 
\begin{equation}
    \frac{d^2\Gamma_{\gamma\gamma}}{d \cos\theta_{12} dE_{\gamma_1}} \propto E_{\gamma_1}^3 E_{\gamma_2}^3 \left( 1 + \frac{4 \alpha_{E1} \chi}{\alpha_{E1}^2 + \chi^2} \cos\theta_{12} +  \cos^2\theta_{12} \right)
\end{equation}
where the angular dependence is decoupled from the energy dependence. Therefore the expected angular correlation is proportional to
\begin{equation} \label{eqn:angular-distribution}
   W(\theta_{12}) \propto \left( 1 + \frac{4 \alpha_{E1} \chi}{\alpha_{E1}^2 + \chi^2} \cos\theta_{12} +  \cos^2\theta_{12} \right)
\end{equation}
neglecting the linear polarization term present in Schirmer \textit{et al.} \cite{schirmer_double_1984}.

%------------------------------------------
\subsection{Room Background}
%------------------------------------------
The measurement of nuclear two-photon decay requires the registration of very rare decay phenomena and thus requires a good understanding of background radiation sources. 
One such background, spurious signals that can mimic the signals of interest, is natural $\gamma$-rays emitted primarily by the decay of three radiative isotopes: potassium-40 ($^{40}$K), uranium-238 ($^{238}$U), and thorium-208 ($^{208}$Th)~\cite[]{aksoy_elemental_1994}.
$^{40}$K decays directly to the stable argon-40 ($^{40}$Ar) emitting a 1460 keV gamma ray, and both $^{238}$U and $^{208}$Th decay through long decay chains eventually terminating in stable lead isotopes. 
Of all the $\gamma$-rays emitted in those long decay chains the most prominent are the 1765 keV and 2615 keV $\gamma$-rays emitted by bismuth-214 ($^{214}$Bi) and thalium-208 ($^{208}$Tl) respectively.

% * could add more statements about where the decays come from. 

\end{document}
%------------------------------------------