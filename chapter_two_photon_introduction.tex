%! Author = cnatzke
%! Date = 05/24/2022
\documentclass[cnatzke_thesis_proposal.tex]{subfiles}
\usepackage{amssymb}
\begin{document}
%\chapter{Introduction}
%%%%%%%%%%%%%%%%%%%%
    \subsection{Nuclear Equation of State}
%%%%%%%%%%%%%%%%%%%%
The Nuclear Equation of State (EOS) governs the properties of nuclear matter and defines the properties of neutron-rich nuclei, the structure of neutron stars, supernova explosions and compact object mergers. \cite{Kaufmann2020} The EOS is well constrainted for symmetric nuclear matter, matter where the number of protons and neutrons are roughly equal, but the properties for neutron rich matter require further investigation. \cite{Danielewicz2002} The properties of interest are encoded in the nuclear symmety enery $S(n)$ and the slope parameter $L$. Atomic nuclei provide a terrestrial system to study the slope parameter using the neutron-skin thickness, the difference between the point-neutron and point-proton radii, mathemtically denoted as $R_{skin} = R_n - R_p$. \cite{Tsang2012} A direct experimental measurement of $R_{skin}$ is extremely challenging and can be extracted by its correlation to collective isovector observables at low energies. \cite{Birkhan2017} One such observable is the dipole polarizability $\alpha_D$ \cite{Birkhan2017}.

%%%%%%%%%%%%%%%%%%%%
    \subsection{Electric Polarizability and Magnetic Susceptibility}
%%%%%%%%%%%%%%%%%%%%
Polarizability is the tendency of matter to create a dipole moment in the presence of an applied external electromagnetic field. For applied electric fields the electric polarizability is a measure of how strong the induced electric field and the magnetic susceptibility is a meausure for the induced magnetic field. \cite{Garg2012} The strength of this response spans many orders of magnitude depending on the scale, for example the magnetic susceptibility of a nucleus is twelve orders of magntitude smaller than that of an atom. \cite{Knupfer1985}

Nuclear DFT calculations suggest a strong correlation between $\alpha_D$ and the neutron radius , $R_{n}$, thereby providing an experimental observable related to the neutron skin thickness.  \cite{Hagen2016}

The electric dipole polarizability has been measured for a variety of stable isotopes, $^{48}$Ca \cite{Birkhan2017}, $^{120}$Sn \cite{Hashimoto2015}, and $^{208}$Pb \cite{Tamii2011} using proton inelastic scattering, and $\alpha_D$ has only been measured for one short-lived nucleus - $^{68}$Ni. \cite{Rossi2013}
At this time the electric dipole polarizability for a short lived nuclear state has not been experimentally measured, but using the \textit{transition} electric polarizabilty, the difference in $\alpha_D$ between the two states, and the ground state polarizabilty the polarizabilty of a short-lived nuclear state can be measured.

The only known experimental probe for the transition electric dipole polarizabilty, $\alpha_D^{fi}$ is nuclear two-photon decay.

%%%%%%%%%%%%%%%%%%%%
\subsection{Nuclear Gamma Decay}
%%%%%%%%%%%%%%%%%%%%
Nuclear gamma decay is an electromagnetic process in which an unstable nucleus deexcites into a more energetically favourable state via the emission of a photon of gamma radiation. The kinetic energy of the photon is equivalent to the energy difference in the initial and final states of the nucleus, save for a small amount of recoil kinetic energy imparted to the nucleus, and the photon's angular momentum equals the difference in angular momentum of the nucleus' state. This restriction on angular momentum forbids the emission of a gamma-ray from two states with equal angular momentum, a photon must have at least 1$\hbar$ unit of angular momentum, denoted as an E0 transition.

Gamma-decay often competes with another first-order decay mode called Internal Conversion. Internal conversion (IC) is an electroweak process where the nucleus deexcites by transferring its energy to an atomic electron. Internal Conversion is particularly important for transitions where both the initial and final states have a $0^+$ spin-parity since IC is the only direct process by which the nucleus can deexcite. \cite{Krane1988}

%%%%%%%%%%%%%%%%%%%%
\subsection{Nuclear Two-photon Decay}
%%%%%%%%%%%%%%%%%%%%

Nuclear two-photon decay is a second-order electromagnetic process wherein the nucleus simultaneously emits two photons of continuous energy that sum to the initial excitation energy. \cite{Kramp1987}

%%%%%%%%%%%%%%%%%%%%
\end{document}
%%%%%%%%%%%%%%%%%%%%
