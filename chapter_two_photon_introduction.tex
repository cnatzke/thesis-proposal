%! Author = cnatzke
%! Date = 05/24/2022
\documentclass[cnatzke_thesis_proposal.tex]{subfiles}
\usepackage{amssymb}
\begin{document}
%\chapter{Introduction}
%%%%%%%%%%%%%%%%%%%%
    \subsection{Nuclear Equation of State}
%%%%%%%%%%%%%%%%%%%%
The Nuclear Equation of State (EOS) governs the properties of nuclear matter and defines the properties of neutron-rich nuclei, the structure of neutron stars, supernova explosions and compact object mergers. \cite{Kaufmann2020} The EOS is well constrainted for symmetric nuclear matter, matter where the number of protons and neutrons are roughly equal, but the properties for neutron rich matter require further investigation. \cite{Danielewicz2002} The properties of interest are encoded in the nuclear symmety enery $S(n)$ and the slope parameter $L$. Atomic nuclei provide a terrestrial system to study the slope parameter using the neutron-skin thickness, the difference between the point-neutron and point-proton radii, mathemtically denoted as $R_{skin} = R_n - R_p$. \cite{Tsang2012} A direct experimental measurement of $R_{skin}$ is extremely challenging and can be extracted by its correlation to collective isovector observables at low energies. \cite{Birkhan2017} One such observable is the dipole polarizability $\alpha_D$ \cite{Birkhan2017}.

%%%%%%%%%%%%%%%%%%%%
    \subsection{Electric Polarizability and Magnetic Susceptibility}
%%%%%%%%%%%%%%%%%%%%
Polarizability is the tendency of matter to create a dipole moment in the presence of an applied external electromagnetic field. For applied electric fields the electric polarizability is a measure of how strong the induced electric field and the magnetic susceptibility is a meausure for the induced magnetic field. 


%%%%%%%%%%%%%%%%%%%%
    \subsection{Nuclear Two-photon Decay}
%%%%%%%%%%%%%%%%%%%%

    Nuclear two-photon decay is a second-order electromagnetic process wherein the nucleus simultaneously emits two photons of continuous energy that sum to the initial excitation energy. \cite{Kramp1987} This decay mo processes such as Internal Conversion (IC) and Electron Capture (EC) are common decay modes dealing with interactions that take place between an atom's nucleus and its bound electrons.
%%%%%%%%%%%%%%%%%%%%
\end{document}
%%%%%%%%%%%%%%%%%%%%
