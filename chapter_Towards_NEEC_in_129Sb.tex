\documentclass[jon_ringuette_thesis_proposal.tex]{subfiles}
\begin{document}

    \chapter{Towards NEEC of $^{129m1}Sb$ in the TITAN EBIT}

    \subsection{$^{129m1}Sb$ As a Candidate For NEEC}
    Stimulation of NEEC has thus far only ever been attempted using stable beam.
    Since this experiment is located at a Rare Isotope Beam (RIB) facility there are vastly more NEEC candidates available to examine.
    An exhaustive search was performed looking for initial NEEC candidates without the stable beam constraints.
    Candidates were examined based on their binding energies, the differences in their isomeric states relative to their inner shell binding energies, along with lifetime and decay paths.
    Further selection criteria were then imposed such as if TRIUMF could produce the ion or isomeric state in question and if there was a reliable $\gamma$-ray decay channel that would yield $\gamma$'s outside of regions with a lot of background noise.
    Following this search several candidates were found, however $^{129m1}Sb$ stood out as an excellent initial candidate due to the first isomeric state having a relatively long half-life of 17.7min, as well as a small difference between the energies of the first and second isomeric state $9.70(8)keV$.
    In addition, the isomers binding energy has the potential for NEEC via either an L (He-like) or M (Ne-like) shell vacancy.

    \begin{figure}[H]
        \begin{center}
            \includegraphics[scale=.27]{sb129_NEECDecayScheme}
        \end{center}
        \caption{\small The low-lying state level scheme for $^{129}$Sb. Shown are the first and second isomeric states at 1851~keV and 1861~keV, respectively, between which the NEEC process will be stimulated. The signal of the NEEC stimulation is the detection of the weakly populated 732~keV, 700~keV, and/or 1161~keV $\gamma$ rays which uniquely follow the de-excitation of the 1861~keV $^{129m2}$Sb state. The lower-right insert shows the $\gamma$ lines that follow the $\beta^-$ decay of $^{129m1}$Sb. \cite{eec_SOP_NEEC}}
        \label{fig:sb129_neec_decay_scheme}
    \end{figure}

    On further examination of the candidate $^{129m1}Sb$ as illustrated in \ref{fig:sb129_neec_decay_scheme} it was found that the higher lying isomeric state (m2) for which NEEC would trigger has a very short half-life ($2.2\mu s$).
    Even though this m2 isomeric state decays via internal conversion (IC) $90\%$ of the time there is a $699.64$keV $\gamma$ released $7\%$ of the time which then causes a secondary decay down to the ground state releasing a $1161.42$keV $\gamma$-ray.
    This secondary $\gamma$ at $1161.42$keV, as will be shown later, is in a particularly clean region of the background spectrum in the EBIT and is in a very optimal region for the HPGe's used on the experimental setup.

    With the identification of an initial candidate NEEC cross-sections were calculated for both the He-like and Ne-like states \cite{palffy2021_private} and are described in \ref{tab:neec_cross_sections}.


    \begin{table}
        \centering
        \begin{tabular}{|l|l|l|}
            \hline
            Electron Vacancy  & NEEC Cross-section           & Electron Energy Required for NEEC \\ \hline
            L-shell (He-like) & $1.6\times 10^{-3}b\cdot eV$ & $\sim 6.95$keV                    \\ \hline
            M-shell (Ne-like) & $3.0\times 10^{-5}b\cdot eV$ & $\sim 1$keV                       \\ \hline
        \end{tabular}
        \caption{\small Cross-sections for the NEEC transition given either a L and M shell vacancy}
        \label{tab:neec_cross_sections}
    \end{table}

    From the above table it is clear that the He-like case would be optimal from strictly a cross-section perspective.
    However, experimental constraints must be evaluated.
    The He-like case poses two main difficulties, the first being that to strip the atom down to the L-shell requires the electron gun to run at $\sim 60$keV, which is within the operating parameters of the EBIT, however it is towards the upper limit of the switching power supplies.
    The second difficulty, and the more important one, comes from the electron energy required for NEEC being only around $1$keV.
    The challenge here is that the electron gun energy must be very rapidly changed between $60$keV (charge breeding) and $1$keV (NEECing).
    Also given the large difference in energies between charge breeding and NEECing has a side effect shown later in simulations that the He-like charge state will very rapidly diminish giving only a brief window to trigger the NEEC reaction.

    The Ne-like case, while having a less desirable NEEC cross-section, only requires the electron gun to run at $6.65$keV to ionize to the desired M-shell and then a small change to $6.95$keV to trigger NEEC.
    It is important to keep in mind here that the NEEC triggering energy of the electron beam in the Ne-like case being $6.95$keV is well below the required energy for the electron to cause the excitation of the isomer to the m2 state in any manner except for NEEC.
    Simulations in Section~\ref{cbsim_simulations} will also show how this greatly increases our NEECing time due to the Ne-like population only falling off by about half as opposed to fully disappearing as in the He-like case.

    \subsection{TITAN EBIT Preparations}

    \subsubsection{EBIT Cycling}
    In order for the EBIT to be used to trigger NEEC there are several steps that must be performed each of which has precise timing requirements.

    \begin{figure}[H]
        \begin{center}
            \includegraphics[scale=.15]{ebitcyclemrtof}
        \end{center}
        \caption{\small Block diagram of complete NEEC cycle for the TITAN setup.}
        \label{fig:ebit_cycle_mrtof}
    \end{figure}

    As outlined in \ref{fig:ebit_cycle_mrtof} and concentrating on the portion within the EBIT there are 4 main parts to a EBIT cycle for stimulating NEEC.

    Step 1 being stacking, this is a processes by which the EBIT's potential field is lowered at the trap entrance to allow singly charged ions in while not lowering it enough that any existing, more highly charged ions, can escape.
    By repeatedly performing this operation it is possible to maximize the number of ions in the EBIT trap up to the space charge limit of the EBIT ($\sim 1\times 10^8$ ions) as illustrated in \ref{fig:ebit_stacking}.

    \begin{figure}[H]
        \begin{center}
            \includegraphics[scale=.6]{MIscheme}
        \end{center}
        \caption{\small Process by which the EBIT can lower one side of the trapping potential well to allow in additional single charged ions while not allowing the highly-charged ions to escape creating a process by which ions can be stacked in the trap to reach the EBIT's space charge limit. \cite{Multi_reflec} \cite{Ebit_klawitter}}
        \label{fig:ebit_stacking}
    \end{figure}

    The reason for this process is due to the space-charge limit of the EBIT being rather large ($\sim 1\times 10^8$ ions) while the space-charge limit of the TITAN (Radio Frequency Quadrupole) RFQ responsible for cooling and bunching the beam is $\sim 1\times 10^6$ ions and that of the MR-TOF-MS being $\sim 1\times 10^5$ ions.
    This process of stacking ions in the trap has been tested by previous groups performing decay spectroscopy experiments with the EBIT \cite{Lennarz2015}.
    % ****Include image of this eventually

    Step 2 in the cycle is charge breeding, this cycle is continuously being performed even during the stacking process.
    By running this concurrently with the stacking process it is ensured that the ions in the trap are reaching a greater than singly charged state causing them to be trapped even when the trap potential is slightly lowered to allow further singly-charged ions in.
    During this charge breeding phase the trapped ions are being bombarded by electrons generated via the 500mA electron gun causing electron-impact ionization.

    Step 3, NEECing, is where NEEC is triggered.
    This step is performed by adjusting the voltage of the associated EBIT power supply to change the electron energy from that required for charge breeding to the required state to that of triggering NEEC (NEECing).

    Step 4, the final step of this process, is to eject the ions from the trapping region of the EBIT and send them towards TITAN's Microchannel Plate (MCP) which can distinguish between charge states that were contained in the EBIT.
    This step is performed by fully lowering the injection side of the potential well which is generated via the cold drift-tubes and simultaneously raising the potential at the center of the trap forcing the highly-charged ions to be evicted from the EBIT.

    Changing between the NEECing and charge-breeding steps in this process is performed by a pre-programmed Arbitrary Function Generator (AFG).
    The AFG is directly tied into high voltage power supplies attached to the collector side EBIT drift tubes, these power supplies receive a low power output waveform (1-5V)  from the AFG which directly controls the high voltage output of the power supplies.
    This configuration can be used to quick ramp the HV power supplies or slowly using any of kind step or sinusoidal function.




    %%%%%%%%%%%%%%%%%%%%%%%%%%%%%%%%%%%%%%%%%%%%%%%%%%%%%%%%%%%%%%%%%%%%%%%%%%%%%%%%%%

    \subsection{Beam Purification Requirements}
    To further drive down background and be able to push further towards the theoretical value of the NEEC process investigations into how to further purify the beam coming from TRIUMF are currently underway.
    The most promising possibility is using the TITAN Multi-Reflection Time Of Flight Mass Spectrometer (MR-TOF-MS) to remove all additional ion contamination.
    This device is capable of separating out isomeric states using the time of flight of the ions within its trapping region \ref{fig:sb129_mrtof_beam_composition}.

    \begin{figure}[H]
        \begin{center}
            \includegraphics[scale=.5]{129 beam composition}
        \end{center}
        \caption{\small Beam composition from MR-TOF-MS $^{129}In$ experiment with lasers set to optimize for $^{129}In$ and not $^{129m1}Sb$ (Provided by TITAN)}
        \label{fig:sb129_mrtof_beam_composition}
    \end{figure}

    As seem above for a TITAN beam-time centered around the measurement of $^{129}In$ the first excited state of $^{129}Sb$ is easily distinguishable both from the other isotopes but also from the ground state of $^{129}Sb$.


    \begin{figure}[H]
        \begin{center}
            \includegraphics[scale=.9]{Principle-of-re-trapping}
        \end{center}
        \caption{\small Re-trapping process by which the MR-TOF-MS is able to separate out different ions based on time of flight. \cite{Excitation1865}}
        \label{fig:mrtof-retrapping_guide}
    \end{figure}

    The MR-TOF-MS can also be used for what is referred to as re-trapping, in this process the injection trap is also used to trap ions of interest from the time of flight trapping region based on their flight time and reject others with slower or faster flight times.
    Using this technique it should be possible to pick out only the ion or isomer of interest.
    This could allow for the beam composition being injected into the EBIT to be pure $^{129m1}Sb$ instead of something similar to what has been previously seen in \ref{fig:sb129_mrtof_beam_composition}.

    \begin{figure}[H]
        \begin{center}
            \includegraphics[scale=.4]{MRtof_retrapping}
        \end{center}
        \caption{\small Re-trapping in the MR-TOF-MS being used to separate out the isomeric states of $^{129}In$ (Provided by TITAN)}
        \label{fig:mrtof_retrapping_In}
    \end{figure}

    This technique has been demonstrated in the past (\ref{fig:mrtof_retrapping_In}) however further investigation is needed into the viability of this technique.
    Towards that end a dataset is currently being analyzed from a recent MR-TOF-MS beam time for $^{35}Mg$.
    From this dataset that has a more optimized MR-TOF-MS resolution than previously possible during the $^{129}In$ beam-time it is thought that limits can be set for how well this technique could work for future NEEC experiments utilizing the EBIT.

   % \begin{figure}[H]
    %  \begin{center}
     %     \rotatebox[origin=c]{90}{\includegraphics[scale=.5]{Mg35_initial}}
      %\end{center}
      %\caption{\small $^{35}Mg$ preliminary data from a small amount of the beam-time to illustrate the MR-TOF-MS ability to separate isotopic states (Provided by TITAN)}
      %\label{fig:mg35_initial}
    %\end{figure}
\end{document}