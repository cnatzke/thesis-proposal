\documentclass[jon_ringuette_thesis.tex]{subfiles}
\usepackage{amssymb}
\begin{document}
%\chapter{Introduction}
    \subsection{Atomic Physics}
    In 1897 J.J.\ Thomson published a groundbreaking paper~\cite{Thomson1897} where using a cathode ray tube he was able to determine that observed 'rays' were in fact negatively charged particles known now as electrons.
    This ground-breaking discovery of a point-like particle dissolved any prior suggestions that the atom was in fact purely constructed from neutrally charged particles and brought about the field of atomic physics.
    Quickly following this discovery Bohr constituted a model for the atom which mimicked the solar system with a positively charged nucleus at the center much like our Sun and negatively charged electrons orbiting around it, bound together not by gravity but by the Coulomb potential (A in Fig. \ref{fig:atom_classic_quantum}).

    \begin{figure}[H]
        \begin{center}
            \includegraphics[scale=.1]{atom_classic_quantum.jpeg}
        \end{center}
        \caption{fA. Classical model of Atom, B. Quantum model of atom \cite{LeBellac2006}}
        \label{fig:atom_classic_quantum}
    \end{figure}

    Following on that initial work it was discovered that electrons do not have have well defined orbits which can be placed at any distance from the nucleus, like planetary bodies, but rather have discrete or quantized energy levels.
    Further, thanks to the Heisenberg uncertainty principle it was shown that we are unable to know both the momentum and position absolutely for a subatomic particle (Eq. \ref{eq:heisenberg_momentum_energy}) , such as the electron, and therefore the orbit of the electron must be interpreted to be a probability cloud (B in Fig. \ref{fig:atom_classic_quantum}).

    \begin{equation}
        \Delta x \Delta p \geqq \frac{\hbar}{2}
        \label{eq:heisenberg_momentum_energy}
    \end{equation}
    The discovery of quantized energy levels for the electrons mentioned above led Bohr to a classical approach with a rudimentary quantum interpretation for the energy levels of the Hydrogen atom (Eq. \ref{eq:bohr_energy_levels}) where $m_e$ is the mass of the electron, $e$ is the charge and $n$ is the principle quantum number as referred to as the shell. \cite{LeBellac2006}.

    \begin{equation}
        E_n = \frac{m_e e^4}{2 \hbar^2 n^2}, n=1, 2, ...
        \label{eq:bohr_energy_levels}
    \end{equation}

    While this proved useful for the Hydrogen atom it was somewhat happenstance that it worked at all and has a number of problems.
    These problems manifested in the inability to determine why some spectral lines were brighter than others, and an inability to calculate transition probabilities.
    In order to accurately calculate the energy levels for Hydrogenic atoms and beyond one needs to make use of the formalisms of quantum mechanics.
    Taking the now traditional approach of applying a Hamiltonian \ref{eq:electron_hamiltonian} for a three dimensional two body system to the Schr\"odinger equation Eq. \ref{eq:schrodinger_eq}

    \begin{equation}
        H \doteq -\frac{\hbar^2}{2\mu} \Delta^2 + V(r)
        \label{eq:electron_hamiltonian}
    \end{equation}

    \begin{equation}
        i\hbar\frac{d}{dt}\ket{\Psi} = H\ket{\Psi}
        \label{eq:schrodinger_eq}
    \end{equation}

    we find a more accurate model for the allowed quantized energy levels for the Hydrogen and Hydrogenic atoms in of the form of Eq. \ref{eq:eigenvalue_hydrogen}.

    \begin{equation}
        E_n = -\frac{1}{2n^2}\left( \frac{Ze^2}{4\pi \epsilon_0} \right)^2 \frac{u}{\hbar^2}, n=1, 2, ...
        \label{eq:eigenvalue_hydrogen}
    \end{equation}

    With these pieces a more accurate quantum interpretation for the nucleus with orbiting electrons has emerged.

    Unlike in the classical universe in the quantum universe in which we live every fundamental particle is identical, for example there is no way to distinguish between any two electrons.
    This postulate of quantum mechanics directly led to the Pauli exclusion principle or the more encompassing symmetrization postulate which in context states that electrons are Fermions which are half-integer spin antisymmetric particles.
    As such no two electrons associated with an atom can share the exact same state.
    The state of an electron can be defined by it's associated four quantum numbers such as spin ($s$), principle quantum number ($n$), angular quantum number ($l$), and magnetic quantum number (m).
    With this information we can start to build up a model for how the electrons around at atom are configured.
    Each allowed energy level which an electron is allowed to occupy is referred to as a shell with each shell having a degeneracy or allowed number of electrons of $2n^2$ with n being the principle quantum number (\ref{tab:atomic_shell}).

    \begin{table}
        \centering
        \caption{Maximum Number of Electrons Per Shell}
        \label{mine}
        \begin{tabular}{|l|l|l|}
            \hline
            n & Shell & Number of Electrons \\
            \hline
            1 & 1st   & 2                   \\
            \hline
            2 & 2nd   & 8                   \\
            \hline
            3 & 3rd   & 18                  \\
            \hline
            4 & 4th   & 32                  \\
            \hline
        \end{tabular}
        \label{tab:atomic_shell}
    \end{table}



    In addition each shell is composed of subshells which allow for $2(2l+1)$ electrons and each shell can have multiple subshells depending on the atomic configuration (\ref{tab:atomic_subshell}).
    \begin{table}
        \centering
        \caption{Maximum Number of Electrons Per Subshell}
        \label{mine}
        \begin{tabular}{|l|l|}
            \hline
            Subshell & Electrons \\
            \hline
            s        & 2         \\
            \hline
            p        & 6         \\
            \hline
            d        & 10        \\
            \hline
            f        & 14        \\
            \hline
        \end{tabular}
        \label{tab:atomic_subshell}
    \end{table}

    Atoms will always fill their electrons from the first shell outwards with the inner most having the highest energy as they are closest to the nucleus.
    There is also a screening or shielding effect that happens due to the inner electrons which prevents the outer electrons from seeing as much charge. This lifts the $l$ degeneracy giving energy levels determined by the $n$ and $l$ quantum numbers.

    \begin{figure}[H]
        \begin{center}
            \includegraphics[scale=.2]{electron_energy_levels.png}
        \end{center}
        \caption{Electron Energy Levels around nucleus}
        \label{fig:atomic_energy_levels}
    \end{figure}
% Probably tie the above together a little better
%Need a level picture, explain how electrons can absorb photons and be knocked between orbits or out entirely because of that..

    With the description above of the different shells and subshells it is possible to use to first order (Eq. \ref{eq:eigenvalue_hydrogen}) to determine the amount of energy required to remove an electron from a particular shell (or principle quantum number) of the atom or more precisely put the energy required to allow the electron to overcome it's potential barrier.
    This commonly happens when an electron undergoes a collision with another particle or is excited by photo stimulation.

%%%%%%%%%%%%%%%%%%%%%%%%%%%%%%%%%%%%%%%%%%%%%%%%%%%%%%%%%%%%%%%%%%%%%%%%%%%%

    \subsection{Nuclear Physics}
    % Pull heavily from Bacca2016 & Honma2002
    % binding energy stuff
    % Harmonic osc. potential?
    % Liquid drop
    % Wood saxon potential
    % Nuclear shell model (interacting and non-interacting)
    % Maybe include fermi's golden rule? (eq 2.79 in krane)
    % \cite{krane}

    Unlike atomic physics where most observations can be fully characterized using the theory of quantum electrodynamics, nuclear physics is governed by several different competing theories and thus is typically addressed in a phenomenological manner \cite{Krane1988}. The basic idea is simple, you take protons and neutrons and glue them together using the strong nuclear force which is 137x greater than that of the electromagnetic.
    However these glued together nuclei are rarely stable and the strong nuclear force is only attractive at very short distances hinting at a deeper complexity at work.
    %%%%%%%%%%%% SUBSUBSECTION %%%%%%%%%%%%%%%

    \subsubsection{Building towards a Shell model}
    To study the nucleus and to start building towards some useful theory, namely the shell model theory, it is important to understand how the shell model evolved. A key component of early models was the nuclear binding energy which is the difference between the particles which make up the nucleus and the total energy of the system (Eq. \ref{eq:nuclear_binding_energy})
    \begin{equation}
        BE(Z, A) = Z m_p c^2 + Nm_{n}c^2 - m_N(Z, N)c^2
        \label{eq:nuclear_binding_energy}
    \end{equation}

    where $m_p$, and $m_n$ are the masses of the proton and neutron respectively, and $m_N(Z, N)$ is the total mass of the nucleus \cite{Bacca2016}.
    In 1935 von Weisz{\"a}cker proposed a simple model based off of how a liquid droplet is modeled, known as the "droplet" model. This model takes into account the nuclear binding energy (Eq.\ref{eq:nuclear_binding_energy}) as well as an enhancement that has a number of adjustable parameters and treats the nucleus as a liquid droplet. This model combining the droplet model (Eq.\ref{eq:binding_energy_droplet}) and nuclear binding energy (Eq.\ref{eq:nuclear_binding_energy}) forms the basis of the semi-empirical mass formula (SEMF).

    \begin{equation}
        BE(Z, A) = a_{vol} A - a_{sur} A^{2/3} - a_{Coul} \frac{Z^2}{A^{1/3}} - a_{asy} \frac{(N-Z)^2}{A} + \delta
        \label{eq:binding_energy_droplet}
    \end{equation}

    In Eq.\ref{eq:binding_energy_droplet} the four leading $a$ terms in order are the volume, surface, Coulomb and asymmetry terms and are fit to experimental data from stable nuclei. The final $\delta$ term is there to take into account a tendency for like nucleons to couple pairwise primarily in stable configurations. This leads to a gain in binding energy when both Z and N are odd due to one of the odd protons converting to a neutron to form a pair. \cite{Krane1988}.
    \begin{figure}[H]
        \begin{center}
            \includegraphics[scale=.5]{binding_energy_vs_A.png}
        \end{center}
        \caption{Binding energy per nucleon with the dots representing experimental data from stable nuclei and the fitted curve showing the liquid drop model \cite{Bacca2016}}
        \label{fig:binding_energy_vs_nucleons}
    \end{figure}

    This model quickly breaks down however in that it does not model the observed behavior in \ref{fig:binding_energy_vs_nucleons} where there is a clearly defined peak structure which hints at a possible shell structure in the nucleus \cite{Bacca2016}.
    To continue down the path of a possible shell like structure to the nucleus let us examine the equation for 2 neutron separation (Eq.\ref{eq:two_neutron_separation}).

    \begin{equation}
        S_{2n} = BE(Z, N) - BE(Z, N - 2)
        \label{eq:two_neutron_separation}
    \end{equation}

    When plotting Eq. \ref{eq:two_neutron_separation} against neutron count in \ref{fig:2_neutron_sep_energies} it is easy to discern a peak structure to the data.
    Each peak can be related to that of a shell closure similar to what is observed in atomic physics for electrons.
    These peak numbers or possible shell closures, where the nuclear binding is particularly strong, are referred to as "magic numbers" and led to to the reframing of the nuclei from a liquid drop model to a more modern nuclear shell model approach.

    \begin{figure}[H]
        \begin{center}
            \includegraphics[scale=.3]{2_neutron_sep_energies.png}
        \end{center}
        \caption{Two neutron separation energies for differing isotopic chains as a function of neutron number relative to the semi-emperical mass formula prediction. Magic numbers in circles above. (Figure from: \cite{Bacca2016}) }
        \label{fig:2_neutron_sep_energies}
    \end{figure}

    To further show the possible shell structure of the nuclei \ref{fig:nuclear_charge_radius} plots the nuclear charge radius as a function of neutron number where again a peaked structure is visable with the same "magic numbers" being observed for the peaks.
    \begin{figure}[H]
        \begin{center}
            \includegraphics[scale=.3]{nuclear_charge_radius.png}
        \end{center}
        \caption{Nuclear charge radii as a function of neutron number with the difference from experimental data vs the liquid drop model. Magic numbers in circles above. (Figure from: \cite{Bacca2016})   }
        \label{fig:nuclear_charge_radius}
    \end{figure}
    %%%%%%%%%%%% SUBSUBSECTION %%%%%%%%%%%%%%%

    \subsubsection{Nuclear Shell Model}
    The first attempts at producing a nuclear shell model started in the 1920s with little success. To develop a better model of the nucleus a quantum mechanical approach was required. For the sake of simplicity and due to the mass of the neucleons only non-relativistic particles will be considered.
    In addition nucleons will be treated as point-like particles which do not require an examination of their constituent components such as quarks and gluons.
    Again we are back to solving the Schr\"odinger equation (Eq. \ref{eq:schrodinger_eq_2})

    \begin{equation}
        H \Psi = E \Psi
        \label{eq:schrodinger_eq_2}
    \end{equation}

    with a Hamiltonian defined as Eq. \ref{eq:hamiltonian}).

    \begin{equation}
        H = K + V = \sum^{A}_i \frac{p^2_i}{2m} + V
        \label{eq:hamiltonian}
    \end{equation}

    It will be assumed that the mass of the nucleons, both protons and neutrons are equal $m_p = m_n$.
    The potential $V$ will need to take into account the strong interaction between nucleons as well as the Coulomb force in the case of the protons.

    Initial models tried a simple approach to the potential in order to try and recreate the observed magic numbers from \ref{fig:2_neutron_sep_energies} and \ref{fig:nuclear_charge_radius} using first a quantum infinite well potential and then a slightly more complicated quantum harmonic oscillator potential (Eq.\ref{eq:harmonic_osc_potential}) \ref{tab:inf_well_harm_osc_magic_numbers}.

    \begin{equation}
        V = \frac{p^2}{2m} + \frac{1}{2} m w^2 r^2
        \label{eq:harmonic_osc_potential}
    \end{equation}


    \begin{table}
        \centering
        \caption{Magic numbers produced through simple quantum potentials}
        \label{tab:inf_well_harm_osc_magic_numbers}
        \begin{tabular}{|l|l|l|}
            \hline
            Infinite Well & Harmonic Oscillator & Measured \\
            \hline
            2             & 2                   & 2        \\
            \hline
            8             & 8                   & 8        \\
            \hline
            20            & 20                  & 20       \\
            \hline
            34            & 40                  & 28       \\
            \hline
            58            & 70                  & 50       \\
            \hline
            92            & 112                 & 82       \\
            \hline
        \end{tabular}
    \end{table}

    The obvious problems with those models with predicting the magic shell numbers indicates that a better potential must be chosen.
    As empirical evidence suggests that the potential should be proportional to the nuclear density of states a Woods-Saxon potential was tried.

    \begin{equation}
        V_{central}(\mathbf{r}) = \frac{-V_0}{1 + exp[(r - R)/a]}
        \label{eq:wood_saxon_potential}
    \end{equation}

    In the Woods-Saxon potential in Eq. \ref{eq:wood_saxon_potential} where $R$ is the mean radius $R=1.25A^{1/3}$, and $a$ is the skin thickness.

    \begin{figure}[H]
        \begin{center}
            \includegraphics[scale=.8]{shell_model_potential.png}
        \end{center}
        \caption{Mean field potential with the y-axis being in MeV, A=50, V0=50, a=0.5fm, R=4.6fm}
        \label{fig:shell_model_potential}
    \end{figure}

    In \ref{fig:shell_model_potential} a potential curve is generated from Eq. \ref{eq:wood_saxon_potential}, this potential manages to capture the first several magic numbers (2, 8, 20), however it then starts to fail much like the other simplistic models as as we go to higher energies.

    \begin{table}
        \centering
        \caption{Magic Numbers: Wood-Saxon potential vs. Measured }
        \label{wood_saxon_vs_measured}
        \begin{tabular}{|l|l|}
            \hline
            Wood-Saxon Potential & Measured \\
            \hline
            2                    & 2        \\
            \hline
            8                    & 8        \\
            \hline
            20                   & 20       \\
            \hline
            40                   & 28       \\
            \hline
            58                   & 50       \\
            \hline
            92                   & 82       \\
            \hline
        \end{tabular}
    \end{table}

    In 1949 Mayer and Jensen suggested a spin-orbit term in the potential analogous to that in atomic physics. Now instead of just a Woods-Saxon potential we get a total potential of the form of Eq.\ref{eq:potential_woods_plus_spin} where $\mathbf{L}$ is the orbital angular momentum operator and $\mathbf{S}$ is the spin angular momentum operator with $V_{ls}(\mathbf{r})$ being an arbitrary function of the radial coordinate.

    \begin{equation}
        V_{total}(\mathbf{r}) = V_{central}(\mathbf{r}) + V_{ls}(\mathbf{r})(\mathbf{L} \cdot \mathbf{S})
        \label{eq:potential_woods_plus_spin}
    \end{equation}

    Combining the spin and angular moment operators to get a total angular moment vector $\mathbf{J}$ via $\mathbf{J} = \mathbf{L} + \mathbf{S}$. Working with the eigenstates of the total angular momentum vector $\mathbf{J}$ it is possible to get an expectation value of $\mathbf{L} \cdot \mathbf{S}$ (Eq.\ref{eq:ls_expectation_value}).

    \begin{equation}
        \langle ls \rangle = \frac{\hbar^2}{2}[j(j+1) - l(l+1) - s(s+1)] = \hbar^2 \left\{
        \begin{array}{ll}
            l/2      & for j = l + \frac{1}{2} \\
            -(l+1)/2 & for j=l-\frac{1}{2}
        \end{array}
        \right.
    \end{equation}

    Due to only dealing with a single nucleon $s = \frac{1}{2}$, finally with this the so called magic numbers come out correctly for this quantum appraoch (see \ref{fig:spin_orbit_potential}).
    \begin{figure}[H]
        \begin{center}
            \includegraphics[scale=0.4]{spin_orbit_potential.png}
        \end{center}
        \caption{}
        \label{fig:spin_orbit_potential}
    \end{figure}

    With the energy spitting between the two levels defined as

    \begin{equation}
        \Delta E_{ls} = \frac{2l+2}{2}\hbar^2 \langle V_{ls \rangle}
        \label{eq:energy_split}
    \end{equation}

    With the addition of the spin-orbit term the shell model is able to predict nuclear magic numbers, spins, and parities of ground state nuclei as well as the pairing term for the semi-empirical mass formula \cite{brmartin}. However it does still have a number of issues. This theory assumes non-interaction between nucleons so while it does a reasonable job of predicting single-particle states in a nuclei near a closed shell it loses some of its predictive power when going away from these closed shell regions. When studying nuclei that are not near these closed shell regions the problem can quickly spiral as it becomes a n-body problem. The Pauli exclusion principle can help as it indicates there is little interaction from closed shells and that these can be treated as a type of inert core. That allows for the calculation of only the "valence" nucleons drastically reducing computation.


%%%%%%%%%%%%%%%%%%%%%%%%%%%%%%%%%%%%%%%%%%%%%%%%%%%%%%%%%%%%%%%%%%%%%%%%%%%%

    \subsection{Bridging between Nuclear and Atomic physics}
    In the previous sections the electrons orbiting the nucleus were dealt with largely independent of the nucleus as well as the nucleus characterized largely independent of the electrons. Often handling them independently is sufficient as the interaction between nucleus and electrons is small for stable atoms. In unstable atoms things get more interesting as there are decay modes that specially make use of this interaction, namely Internal Conversion (IT), and Electron Capture (EC). As NEEC can be considered as the time reverse process of IT it is worth exploring these further.

    \subsubsection{Hyperfine Interaction}
    The Hyperfine interaction is used to describe the delicate interaction between the electrons and the nucleus leading to small shifts in the degenerate energy levels as well as splitting of those energy levels. This comes from higher order electromagnetic multipole moments of the nucleus. The electric monopole moment is not considered here as it is already addressed in the Coulomb interaction. The most promininent of these interactions is that of the interaction between the magnetic moment of the nucleus and the internal magnetic fields in the atom which are caused by the electron's orbital motion and by the electron's spin magnetic moment. \cite{mcintyre}  To describe the intrinsic magnetic momnt of the electron aossociated with its spin we use Eq. \ref{eq:electron_spin_b_moment}.
    \begin{equation}
        \mathbf{\mu}_e = -g_e \frac{e}{2m_e}\mathbf{S} = -g_e \mu_{B} \frac{\mathbf{S}}{\hbar}
        \label{eq:electron_spin_b_moment}
    \end{equation}

    Here $g_e$ is the gyromagnetic ratio (approximately 2) and $\mu_B = e\hbar / 2m_e$ is known as the Bohr magneton. Using $\mathbf{I}$ to represent the proton or general nuclear spin the equation for the magnetic moment of the proton can be represented as follows.

    \begin{equation}
        \mathbf{\mu}_p = g_p \frac{e}{2m_p}\mathbf{I} = g_p \mu_{N} \frac{\mathbf{I}}{\hbar}
        \label{eq:proton_spin_b_moment}
    \end{equation}

    where $\mu_N = e\hbar/2m_p$ is the nuclear magneton, also $g_p$ the gyromagnetic ratio of the proton is typically given as 5.59. This gyromagnetic ratio value comes about due to the compostie quark structure of the proton.

    From these equations it is possible to derive a hyperfine Hamiltonian that can be used to calculate the effect of this interaction for various atoms (Eq\ref{eq:hamiltonian_hyperfine}).

    \begin{equation}
        \mathbf{H'}_{hf} = \mathbf{\mu}_p \cdot \frac{\mu_{0}}{4\pi} \frac{e\mathbf{L}}{mr^3} +
        \frac{\mu_0}{4\pi} \frac{1}{r^3} \left[ \mathbf{\mu}_e \cdot \mathbf{\mu_p} -
        3 \frac{(\mathbf{\mu_e} \cdot \mathbf{r})((\mathbf{\mu_p} \cdot \mathbf{r})}{r^2} \right] -
        \frac{\mu_0}{4\pi} \frac{8\pi}{3} \mathbf{\mu_e} \cdot \mathbf{\mu_p} \delta(\mathbf{r})
        \label{eq:hamiltonian_hyperfine}
    \end{equation}
    In Eq.\ref{eq:hamiltonian_hyperfine} the first term represents the interaction between the proton magnetic moment and the magnetic field from the electron's orbital angular momentum. The next term represents the interaction between two magnetic dipoles when $r \ne 0$. Followed by the last term is consistent with a dipole-dipole interaction and is known as the Fermi contact interaction.

    \subsubsection{Electron Capture}
    Electron capture is a process resulting from the weak or beta-decay interaction between nucleons and leptons (electrons, neutrinos). This process can be represented by Eq.\ref{eq:internal_conversion} where a proton and an electron come together to form a neutron and an electron neutrino. \cite{crasemann}

    \begin{equation}
        p + e^{-} \to n + \nu_e
        \label{eq:internal_conversion}
    \end{equation}

    For this process to occur the atomic mass of the system $M(Z, A; Z)$ but be greater than $M(Z-1, A; Z-1)$ with all the electron orbitals properly filled forming a neutral atom. Note that this is not possible for a Hydrogen atom as $Q_{EC}$ would be less than 0. This processes releases an amount of energy equal to $Q_{EC}$ in Eq.\ref{eq:internal_conversion_energy}.

    \begin{equation}
        Q_{EC} = [M(Z, A; Z) = M(Z-1, A; Z-1)]c^2
        \label{eq:internal_conversion_energy}
    \end{equation}

    The above energy manifests in the following forms: outgoing neutrino, atomic excitation of the daughter system, recoil energy, and possibly nuclear excitation of the daughter system.
    If the daughter system is in an excited state after the transition then typically it will de-excite and in the process releasing a gamma-ray.

    As this process is driven by the weak nuclear force which is a very short range force it is expected that inner electrons (K, L, M)-shells have a much higher chance of participating in electron capture due to their higher density near the nucleus. Also this should not be thought of as a process that only impacts an inner electron and the nucleus, but rather due to the change in the radial scale, can have an impact on the electronic eigenfunctions of the initial and final states leading to a inner-shell hole at a different location after the decay. This new location hole location is typically due to an outer electron filling in an inner shell hole which then also releases an x-ray.

    The leptons can also carry away orbital angular momentum. If the transition causes the lepton to have 0 angular momentum then it is referred to as an allowed transition with laptops that contain a single unit of orbital angular momentum referred to ass first forbidden. To see all the selection rules for nuclear total angular momentum transfer along with partiy change refer to table \ref{tab:forbiddenness_classification}.
    \begin{table}
        \centering
        \caption{Forbidenness Classification of Electron-Capture Transitions}
        \label{tab:forbiddenness_classification}
        \begin{tabular}{|l|l|l|}
            \hline
            & Nuclear Angular Momentum Transfer & Parity Change          \\
            \hline
            Allowed                   & 0, 1                              & No                     \\
            \hline
            First Forbidden           & 0, 1, 2                           & Yes                    \\
            \hline
            Second Forbidden          & 2, 3                              & No                     \\
            \hline
            Nth Forbidden $(2 \le N)$ & N, N+1                            & $\pi_i \pi_f = (-1)^N$ \\
            \hline
        \end{tabular}
    \end{table}


% Electron capture rates...
    The rate in which electron capture will happen takes place can be calculated using Eq.\ref{eq:electron_capture_rate}. Here $g$ is the weak interaction coupling constant, $x$ referes to the various bound atomic orbitals $x=n\kappa$, $n_x$ is the occupation probability of orbital $x$ in the initial atomic state with $n_x = 1$ being when the orbital is full, $C_x$ referes to electron-capture equivilent of a beta-spectrum shape factor and finally $f_x$ is a Fermi function (see Eq.\ref{eq:electron_capture_rate_fx}).

    \begin{equation}
        \lambda = (g^2/2\pi^3) \Sigma_x n_x C_x f_x
        \label{eq:electron_capture_rate}
    \end{equation}

    For the Fermi function in Eq.\ref{eq:electron_capture_rate_fx} $q_x$ represents the momentum of the emitted neutrino when there is an electron hole left in orbital $x$, $\beta_x$ is the wave-function amplitude for an electron in orbital x and finally $B_x$ is the exchange and overlap factor. This factor accounts for the effects of the electron exchange and an imperfect wave-function overlap. \cite{crasemann}

    \begin{equation}
        f_x = (\pi/2)q_x^2 \beta_x^2 B_x
        \label{eq:electron_capture_rate_fx}
    \end{equation}

    \subsubsection{Internal Conversion}
    Internal Conversion (IC) is the process by which an excited nucleus can remove some or all of its extra energy moving to a lower energetic state.  This energy removal is facilitated by an electomagnetic interaction between the nucleus and an electron.  The processes heavily favors inner shell electron (K, L, M, N) and falls off for higher shell numbers (n) as $1\n^3$.  Thi This should not be thought of as a two step processes where by a photon is exchanged with the electron but rather a direct transfer due to the electron and the neutron occasionally having overlapping wave functions.  When the electron is ejected its energy is not that of a spectrum but rather at specific energy levels.  This makes it easy to distinguish electrons emitted via IC rather than $\beta$ decay as the $\beta$ decay electrons will have a continuum of energies These energy levels directly correspond to the amount of energy that the nucleus is releasing during its isomeric transition minus the binding energy of the electron Eq.\ref{eq:ic_energy}.  Where $T_e$ is the kinetic energy instilled in the electron as it escapes, $\Delta E$ is the energy change of the nucleus and B represents the binding energy of the electron.

    \begin{equation}
        T_e = \Delta E -B
        \label{eq:ic_energy}
    \end{equation}

    Again, much like in the Electron Capture process when an inner shell electron is displaced then the higher shell electrons will move to fill the hole whereby a cascade release of x-rays is possible.  IC does compete with $\gamma$ emissions for de-excitation of the nucleus and both must be factored in when computing decay rates of nuclear excited states (Eq.\ref{decay_ic_and_gamma}) where $\lambda_t$ represents the total decay rate with $\lambda_{\gamma}$ and $\lambda_e$ representing the rate due to $\gamma$ emission and IC respectively.
    \begin{equation}
        \lambda_t = \lambda_{\gamma} + \lambda_e
        \label{eq:decay_ic_and_gamma}
    \end{equation}

    The processes of internal conversion competes with $\gamma$ emissions and in comparing the processes it is helpful to consider the ratio between the two competing processes (Eq.\ref{eq:ic_vs_gamma}) where $\alpha$ is the internal conversion coefficient.

    \begin{equation}
        \alpha = \frac{\lambda_e}{\lambda_{\gamma}}
        \label{eq:ic_vs_gamma}
    \end{equation}

    Treating the electrons relativistically the internal conversion coefficient can be approximated for both the (E) electric mutipole and (M) magnetic multipole via Eqs \ref{eq:electric_ic_coeff} and \ref{eq:mag_ic_coeff} where Z is the proton count, and L is the multipolarity.  From these it is shown that the internal conversion coefficient favoring IC happens with heavier nucleui and proportionally increases as $Z^3$.  However as the energy the nucleus is trying to give up increases $\gamma$ emissions become more favorable decreasing the relative decay rates between the two processes by roughly $Energy^{-2.5}$ in favor of $\gamma$ emission.  Also as a higher electron shell number (n) is considered the conversion coefficient falls as $1\n^3$. \cite{Krane1988} \cite{crasemann}
    \begin{equation}
        \alpha(EL) \cong \frac{Z^3}{n_3} \left(  \frac{L}{L + 1} \right) \left( \frac{e^2}{4\pi \epsilon_0 \hbar c} \right)^4 \left(\frac{2m_e c^2}{Energy} \right)^{L+5/2}
        \label{eq:electric_ic_coeff}
    \end{equation}

    \begin{equation}
        \alpha(EM) \cong \frac{Z^3}{n_3}  \left( \frac{e^2}{4\pi \epsilon_0 \hbar c} \right)^4 \left(\frac{2m_e c^2}{Energy} \right)^{L+3/2}
        \label{eq:mag_ic_coeff}
    \end{equation}
%%%%%%%%%%%%%%%%%%%%%%%%%%%%%%%%%%%%%%%%%%%%%%%%%%%%%%%%%%%%%%%%%%%%%%%%%%%%

    \subsection{Nuclear Excitation via Electron Capture (NEEC)}

    \subsubsection{Little history (verbose}

    \subsubsection{Motivation (battery and astro physics)}

    \subsubsection{Theory}

    \subsubsection{Current State}


\end{document}