\documentclass[cnatzke_thesis_proposal.tex]{subfiles}
\begin{document}

    \begin{abstract}

The nuclear equation of state requires rigorous bench-marking of experimentally observable properties which can be accurately calculated. 
One such observable is the electric polarizability of nuclear matter and its difference for an excited nuclear state. 
One possible way of extracting this quantity is through the second order electromagnetic process of nuclear two-photon emission. 
Searching for this decay mode in systems where the ground and first excited states have $0^+$ spin/parity provides significant experimental advantages since single photon emission is forbidden.
The first excited state of $^{90}$Zr satisfies these conditions, has been previously observed to undergo two-photon emission and is accessible through a $^{90}$Sr source.
The GRIFFIN spectrometer at TRIUMF-ISAC is a powerful set-up for decay studies that has the angular sensitivity, energy resolution, and data acquisition system required to make a precision measurement of $^{90}$Zr decay using a high-activity $^{90}$Sr source as a proof-of-concept measurement.
The techniques developed will be applied to a $^{72}$Ga radioactive ion beam dataset to pursue a novel observation of two-photon decay in $^{72}$Ge.

This document will provide current progress on the measurement of two-photon decay from a $^{90}$Sr source and how the analysis will be extended to $^{72}$Ga datasets that have already been collected.

    \end{abstract}

\end{document}