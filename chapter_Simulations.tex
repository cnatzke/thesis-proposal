\documentclass[jon_ringuette_thesis_proposal.tex]{subfiles}
\begin{document}

    \chapter{Simulations}

    \subsection{GEANT4 - $\gamma$-ray Background}
    In order to determine the viability of the experiment to observe the triggering of NEEC a number of simulations were required.
    One of the larger pieces of this was to put together a simulation of the EBIT and the HPGe detectors in GEANT4 \cite{Agostinelli2003}.
    This enables the simulation of the background within the trapping region as well as simulating the beam composition with all the decay paths therein.
    This combined with several other simulations give a good basis for determining if the experiment is capable of observing the triggering of NEEC.
    \begin{figure}[H]
        \begin{center}

            \includegraphics[scale=0.3]{g4_ebit_trap_center.png}
            %\includegraphics[scale=0.6]{ebit_8pi_geant4_correct.png}
        \end{center}
        \caption{\small GEANT4 simulation of EBIT surrounded by 7 HPGe detectors.}
        \label{fig:g4_ebit_8pis}
    \end{figure}

    The central region in \ref{fig:g4_ebit_8pis} containing all known geometry and material layers within the EBIT.
    Going outwards from there we observe the ports and beryllium windows before finally reaching the HPGe detectors.
    The geometry and material files for the HPGe detectors themselves was directly supplied by Ortec.

    \begin{figure}[H]
        \begin{center}
            \includegraphics[scale=0.45]{8pi_eff_0_2MeV.png}
        \end{center}
        \caption{\small GEANT4 simulation of the HPGe detector array efficiency over $\gamma$-ray energy range. Simulation done with 1 million $\gamma$-ray events at different energies to compute efficiency.}
        \label{fig:g4_efficiency_8pi}
    \end{figure}

    Given the highly detailed GEANT4 model of the EBIT and HPGe detectors it was possible to generate an efficiency curve for the entire HPGe array given 7 detectors.
    This was handled by generating $10^6$ mono-energetic $\gamma$-rays at the center of the simulated EBIT using 10keV steps between 10keV and 2MeV.
    Each of these one million mono-energetic $\gamma$-rays was run through GEANT4 to generate a spectrum.
    Taking into account only occurrences within a few percent of the energy region it was possible to generate an efficiency curve for the entire array (\ref{fig:g4_efficiency_8pi}).
    This could then be used as a basis for determining the array efficiency in any region of interest for further simulations.

    % \begin{figure}[H]
    %    \begin{center}
    %       \includegraphics[scale=0.45]{sb129m1_bg_only_cascade_60s_2e5_1hr_scaled_to_7day.png}
    %  \end{center}
    % \caption{$^{129m1}Sb$ *REPLACE ME* Simulated gamma background in GEANT4. Representing 60s of trapping time over 7 days. Includes %$^{129m1}Sb$ and $^{129}Sb$ decays ** ADD Te!! **}
    %      \label{fig:geant4_cascade_background_7days}
    % \end{figure}
%%%%%%
    \subsection{\label{cbsim_simulations}CBSim - Charge Breeding in EBIT}
    Based on previous work \cite{Macdonald2014},\cite{Currell2005} a Charge Breeding (CB) simulation package was developed in Python (pyCBSIM) \cite{pyCBSIM}.
    As of this writing the simulation accounts for first order effects from electron impact ionization (EI), radiative recombination (RR), and charge exchange (CX).
    %  electron impact ionization (EI) - Je represents the current density of electrons, $\sigma_{i}$ is  the cross section, and f(e,i) an electron-ion overlap factor

    \begin{equation}
        E_i = \frac{J_e}{e}  N_{i}  \sigma_{i}  f_{e, i}
        \label{eq:electron_impact_ionization}
    \end{equation}

    For electron impact ionization in (Eq. \ref{eq:electron_impact_ionization}) we calculate the rate at which electrons will be stripped from ion $i$ via electron collision with the ion. $J_e$ represents the current density of electrons $e n_e v_e$, $\sigma_{i}$ is the cross section, $N_i$ is ions per length in trap and $f_{e, i}$ an electron-ion overlap factor.
    The equation for radiative recombination is essentially the same as (Eq. \ref{eq:electron_impact_ionization}) however the cross-section $\sigma_{i}$ is calculated against the likelihood of the free electrons recombining with the ions.

    \begin{equation}
        f_{e, i} = \frac{N^{in}_{i}}{N_i}
        \label{eq:electron_ion_overlap}
    \end{equation}

    The electron-ion overlap factor (Eq. \ref{eq:electron_ion_overlap}) $i$ is the species, with $N^{in}_{i}$ representing the total number of ions in the beam.

    \begin{equation}
        C_i = v_{i_{avg}} N_0 N_i \sigma_{i}
        \label{eq:charge_exchange}
    \end{equation}

    For the charge exchange rate (Eq. \ref{eq:charge_exchange}) we represent the average speed of the ion based on a Maxwellian speed distribution as $v_{i_{avg}}$, the number density of the background gas $N_0$, and the number density of ions as $N_i$.
    For further reading on calculating cross-sections please see Ref. \cite{Currell2005} .

    pyCBSIM takes in parameters such as electron beam radius, electron beam potential energy, pressure inside of trap, electron beam current, and time the ions are exposed to the electron beam.
    Given these parameters along with specified ions composition of the trap and desired charge states to observe the software is approximates the population of charge states within the trap.
    \begin{figure}[H]
        \centering
        \subfloat[]{
            \includegraphics[width=0.5\textwidth]{sb129_6650_500ma_1sec}
        }
        \subfloat[]{
            \includegraphics[width=0.5\textwidth]{sb129_6650_neec_500ma_1sec}
        }
        \hspace{0mm}
        \subfloat[]{
            \includegraphics[width=0.5\textwidth]{sb129_He_60k_500ma_5sec}
        }
        \subfloat[]{
            \includegraphics[width=0.5\textwidth]{sb129_He_60k_500ma_5sec_neec}
        }
        \caption{\small These 4 diagrams represent charge breeding $^{129m1}Sb$ to both the Ne-like and He-like states based on pyCBSim. The two graphs on the left are simulations of purely charge breeding the two states while those on the right are charge bread until optimal population is obtained where in the beam energy is changed to the NEECing energy.}
        \label{fig:sb129_cb_sim}
    \end{figure}

    It is also capable of varying the electron gun potential energy at different times in order to simulate different experimental situations such as changing between a charge breeding energy and a NEEC triggering energy as is done on the right side of \ref{fig:sb129_cb_sim}.
    In the Ne-like like case it was set to use a 0.5A gun with a breeding potential energy set at 6650eV for threshold charge breeding of the Neon like state of $^{129}Sb$.
    As seen in \ref{fig:sb129_cb_sim} the charge state for both the Ne-like and He-like cases were simulated based on known EBIT parameters.
    These simulations show that in the Ne-like case the charge state is maintained even when the electron gun energy is altered to trigger NEEC giving a large window of opportunity to observe the triggering of NEEC.

    \subsection{Optimizations}\label{subsec:optimizations}

    Simulations above indicate that the optimal breeding time of $^{129m1}Sb$ is around 0.69s.
    Any stacking that is performed within the EBIT should take this into account to ensure that the optimal charge state has been reached prior to injecting more ions into the trap.
    Further investigation into how many bunches to load into the EBIT before triggering NEEC has turned out to be highly depended on how beam purification happens.
    If the beam is pure enough, meaning mostly $^{129m1}Sb$ then loading it in straight through the RFQ is optimal, the RFQ has a space charge limit of around $1\times 10^6$ ions and to fully fill the EBIT would require 20 cycles of stacking.
    However for the initial simulations 10 cycles was chosen due to $1\times 10^8$ being the absolute theoretical maximum that the EBIT trap can hold and so a more cautious $1 \times 10^7$ space charge limit is being assumed.

    \subsection{Estimation of NEEC in $^{129m1}Sb$ Based on Simulations}
    \begin{figure}[H]
        \begin{center}
            \includegraphics[scale=0.4]{neec_with_total_bg_2days_1e7pps_500mA.png}
        \end{center}
        \caption{\small GEANT4 simulation of 2 days worth of beam time assuming a beam composition previously seen during an MR-TOF experiment. This includes background and decay chains.}
        \label{fig:neec_2day_500ma_simulation}
    \end{figure}

    Combining simulations for charge stacking, charge breeding, and the GEANT4 simulation of the detector array over 2 days of beam time it is easy to see from \ref{fig:neec_2day_500ma_simulation} that even in this mildly optimized case the experiment should be able to probe 3 orders of magnitude closer to NEEC theory than the Argonne lab experiment.
    This simulation also includes actual background measurements taken by the HPGe's scaled to that of a 2 day experiment as well as all the primary decay chains for $^{129m1}Sb$ (see \ref{fig:sb129m1_decays}) but all other decays based on beam composition estimates.
    This was generated using 10 stacking cycles into the EBIT directly from the RFQ with a 1s breeding time before each stacking session.
    These 10 stacking cycles were then following by 1min of NEECing before ejecting the ions from the trap and starting over with another 10 stacking cycles.


    \begin{figure}[H]
        \begin{center}
            \includegraphics[scale=.2]{sb129m1_decays_smaller.png}
        \end{center}
        \caption{$^{129m1}Sb$ decay chain}
        \label{fig:sb129m1_decays}
    \end{figure}

    % ** REFER TO EXPTERIMENTAL SETUP AND THESHOLD CHARGE BREEDING**
    % **MAKE INCLUDE ONLY CHARGE BREEDING THEN WITH NEEC ENERGY**
\end{document}