\documentclass[cnatzke_thesis_proposal.tex]{subfiles}
\begin{document}

\chapter{Experimental Methods}

%------------------------------------------
\subsection{Data Sorting}
%------------------------------------------
GRIFFIN data is written to disk according to the GRIFFIN event file format, a form of the general MIDAS file format (see Section~\ref{sec:data_acquisition}).
To properly analyze the data files they must first be unpacked using the analysis package GRSISort~\cite{bildstein_griffincollaborationgrsisort_2019}, a framework built on top of ROOT~\cite{brun_root-projectroot_2019}.
GRSISort converts the MIDAS files to ROOT files via a two stage process.
During the first unpacking stage GRSISort produces a fragment tree, a ROOT file containing time-ordered hits, then by applying an event-building algorithm GRSISort stores hits occurring within a user-specified time coincidence together in another ROOT file called an analysis tree.
In this work a 1 $\mu$s build window, the time-coincidence condition mentioned previously, was used.
The analysis trees can then be sorted into histograms via GRSISort or a custom sorting program compiled against the GRSISort libraries.

%------------------------------------------
\subsection{Detector Calibrations and Performance}
%------------------------------------------
Before any investigation into two-photon decay detection can be performed the data had to be calibrated across a broad energy region. This section will detail the energy calibration process needed for proper analysis of GRIFFIN data. 

%------------------------------------------
\subsubsection{GRIFFIN Energy Calibrations}
%------------------------------------------
To properly take advantage of GRIFFIN's high efficiency and large solid angle coverage data from the HPGe crystals are summed together throughout the analysis.
This summation of data requires a careful energy calibration to preserve high peak resolution of the detector array when summing data together from different crystals. 
The standard method for energy calibration uses known source data to calibrate the energy spectra for the crystals making up the sixteen clovers used in this experiment. 

The three different data collections, performed in 2017, 2018, and 2019 respectively; used two different hardware revisions of the GRIF-16 digitizers. 
The data collected in 2017 and 2018 used revision 1 and the data from 2019 used revision 2. 
Revision 2 significantly increased the performance and stability of the digitizers, removing time-dependent energy nonlinearities, requiring different energy calibration processes between the three data sets. 

%------------------------------------------
\subsubsection{Sum Spectrum}
%------------------------------------------

The two photons released during a two-photon decay event sum to the total energy of the transition, meaning the full energy of the transition should show up as a peak in a sum spectrum. 
A sum spectrum is generated by summing the energies of two incident photons together and recording their resultant sum in a histogram.
This is similar to the GRIFFIN addback algorithm, where the energy of all events detected within a single clover are summed together, but does not restrict the sum to singles events within the same clover; but rather, sums events together that were registered in different crystals regardless of if they were in the same clover of not. 


%------------------------------------------
\end{document}
