\documentclass[cnatzke_thesis_proposal.tex]{subfiles}
\begin{document}

\chapter{Experimental Methods}

%------------------------------------------
\subsection{Data Sorting}
%------------------------------------------
GRIFFIN data is written to disk according to the GRIFFIN event file format, a form of the general MIDAS file format (see Section~\ref{sec:data_acquisition}).
To properly analyze the data files they must first be unpacked using the analysis package GRSISort~\cite{bildstein_griffincollaborationgrsisort_2019}, a framework built on top of ROOT~\cite{brun_root-projectroot_2019}.
GRSISort converts the MIDAS files to ROOT files via a two stage process.
During the first unpacking stage GRSISort produces a fragment tree, a ROOT file containing time-ordered hits, then by applying an event-building algorithm GRSISort stores hits occurring within a user-specified time coincidence together in another ROOT file called an analysis tree.
In this work a 1 $\mu$s build window, the time-coincidence condition mentioned previously, was used.
The analysis trees can then be sorted into histograms via GRSISort or a custom sorting program compiled against the GRSISort libraries.

%------------------------------------------
\subsection{Data Sorting}
%------------------------------------------





%------------------------------------------
\end{document}
